\chapter*{General conclusions and outlooks}
\markboth{General conclusions and outlooks}{}
\addstarredchapter{General conclusions and outlooks}

% RESTATE THESIS STATEMENT
% ========================
The main aim of this thesis has been to make realistic predictions and to 
investigate the sources of uncertainties in the observables of nonaccreting 
cold NS and warm PNS, using a unified metamodeling 
approach for the description of stellar matter energetics, which allows to 
account for the present day constraints provided by nuclear experiments and 
astrophysical observations.
\\

% KEY POINTS
% ==========
We have considered a unified metamodeling approach in order to 
calculate the composition and EoS of cold nonaccreting NS for any 
functional of NM using the associated empirical parameters as inputs, with an 
error which, in the representative case of the SLy4 EoS, is less than $1\%$. 
To model the energy of clusters in the inner crust, we have proposed a version 
of the well-known CLDM based on the metamodeling technique, with a
parametrization of the surface tension suggested from TF calculations at 
extreme isospin ratios. The ground-state composition obtained follows closely 
the ETF results reported in the literature. The main drawback of the 
CLD approach is that quantum effects are lost, yet we have shown that magic 
numbers in the inner crust can be recovered by adding Strutinsky shell 
corrections to the CLD energy, leading to results in very good agreement with 
ETFSI calculations. 
The same sequence of nonspherical pasta phases in the deepest layers of the 
inner crust has been observed for all CLDM considered, yet we have stressed 
that it is sensitive to the behavior of the surface tension at high isospin,
which remains poorly constrained to the present day. 
The uncertainty on the EoS induced by the treatment of the surface energy can
be estimated to the order $10\%$, which corresponds to the difference between
our results with the SLy4 functional and the DH EoS which uses the same
functional but a different treatment of the surface energy. 
The same is true concerning the location of the transition point to 
homogeneous $npe$ matter, which has been computed for several nuclear models
from the crust side. 
We have shown that the anticorrelation between the slope of the symmetry energy
at saturation $L_{sym}$ and the CC transition density, reported in many
previous works, is obtained if the surface energy is only optimized on a
limited set of spherical magic and semimagic nuclei. Conversely, if the surface
tension is optimized on a large pool of data and curvature terms are added,
this correlation disappears and the sensitivity to the EoS is limited to the
poorly known high-order empirical parameters in the isovector sector, 
particularly the skewness $Q_{sym}$.

We have exploited the main asset of the metamodeling technique, namely the 
fact that no artificial correlations are introduced \textit{a priori} among the 
empirical parameters, also to carry out the Bayesian determination of the 
empirical parameters, leading to realistic predictions for NS observables. We 
have considered a flat prior for the empirical parameters whose boundaries are
compatible with experimental constraints. 
The likelihood function that we have constructed takes into account recent 
chiral EFT calculations for SNM and PNM up to $0.20$ fm$^{-3}$, and 
the maximum mass constraint as well as basic physical requirements. 
It also includes a probability in which is encoded the ability of the CLDM to 
fit the masses of AME2016. We have shown that imposing the constraints
associated to the \textit{ab initio} calculations is very effective in 
constraining the empirical parameters in the isovector channel, and that it 
yields correlations among the symmetry energy derivatives at saturation.
We have made general predictions for the static properties using the posterior 
distribution of empirical parameters and found that our results are compatible 
with constraints of the LIGO and Virgo collaborations inferred from GW170817.
%
{Since the only hypothesis of the metamodeling technique is the 
  possibility of expanding the EoS into a power series, this result implies 
  that we have no compelling evidence of a first order phase transition in the 
  core of NS, though the latter cannot of course be excluded.}
%
The fractional crust moment of inertia, which is strongly correlated with the 
location of the CC transition point, has been computed, allowing us to exclude 
a full crustal origin of pulsar glitches if we consider the largest present 
estimation of crustal entrainment. This opens interesting possibilities of the
relevance of superfluid components in the core, particularly in the $^3P_2$
channel for $nn$ and $pp$ pairs.

Following a recent work on the outer crust, we have considered a full
statistical equilibrium of ions in the crust at finite temperature allowing for 
the presence of dripped neutrons. We have evaluated the abundancies of odd-mass 
and odd-charge nuclei present in the outer crust at crystallization. Their
presence is of interest, because it could be the cause for ferromagnetic phase 
transitions. 
We have considered a temperature dependence of protons shell corrections in the 
regime of the inner crust. The crystallization temperature and associated
composition have been calculated in the OCP approximation, and our results
suggest that the highest source of model dependence comes from the smooth part
of the nuclear functional, the most important ingredient to be settled for a
quantitative prediction of the inner crust properties being the surface tension
at extreme isospin ratios. 
Deviations from CCM have been observed at low density and crystallization
temperature, thus potentially having an impact on the simulation of 
$r$-processes.
Finally, we have consistently calculated the impurity parameter in the inner 
crust at crystallization for four BSk functionals that span a relative large 
range in the symmetry energy parameters. This is the first calculation to date
of the impurity factor with nuclear realistic functionals, and it has shown 
that contributions of impurities is nonnegligible, thus potentially altering 
transport properties in the NS crust.
\\

% IMPACT AND FUTURE RESEARCH
% ==========================
Clearly, higher precision in the experimental determination of high-order
isovector empirical parameters in the low-density EFT theoretical predictions 
and in the microscopic modeling of the surface energy at extreme isospin ratios 
are needed to reduce the uncertainties of crustal observables. Moreover, 
we have shown that constraints on dense matter properties can be inferred from 
astrophysical observations within the Bayesian framework. 
Many new measurements are expected in the near future from NICER and LIGO/Virgo 
collaborations, thus we can hope to reduce the uncertainties on the 
high-order symmetry energy derivatives and on NS observables using Bayesian 
inference or machine learning algorithms.

Various applications of the formalism introduced for the description of a MCP 
in a full statistical equilibrium can be envisaged in the future. For instance, 
one could investigate the presence of hyperons in the crust at crystallization. 
Nonspherical geometries can also be considered in the MCP treatment, allowing 
the evaluation of the impurity parameter in the bottom layers of the inner 
crust, and thus verifying the hypothesis of~\cite{Pons2013} that the presence
of a highly resistive layer in the inner crust, might lead to a higher limit to
the spinning period of x-ray pulsars.

During the thesis, I have written an open-source C library, NSEoS, with the 
aim of providing useful tools related to the physics of NS~\cite{NSEoS}. The 
library has been used to produce all the results presented in this thesis.
Moreover, it was already used by different students to perform their internship
under the joint supervision of my thesis advisor and myself, and 
we believe that it can be used as a basis for future studies.

%% KEY POINTS (old; replaced because boring)
%% =========================================
%In Chapter 1, we have determined the composition of the outer 
%crust, inner crust, and core within the CCM hypothesis, and considered a 
%unified metamodeling of the EoS.
%%
%The ground-state composition and EoS of the outer crust at $T=0$ K have been 
%evaluated by the application of the BPS method~\cite{BPS}, using 
%state-of-the-art microscopic HFB theoretical mass tables~\cite{Samyn2002} as 
%a supplement to the present day knowledge on experimental 
%masses~\cite{Huang2017,Welker2017}, when equilibrium nuclei are too neutron 
%rich to be synthesized in the laboratory, that is from 
%$n_B \approx 3\times 10^{-5}$ fm$^{-3}$. 
%All mass models that we have considered agree on the persistence of $N=82$ in 
%the bottom layers of the outer crust.
%%
%In the regime of the inner crust, the ground-state is fully model dependent due 
%to the presence of a neutron gas surrounding the clusters. 
%We have proposed a version of the well-known CLDM, based on the metamodeling 
%technique introduced in~\cite{Margueron2018a,Margueron2018b}, to model the
%cluster energetics. The inputs of our CLDM are the so-called empirical 
%parameters, corresponding to the successive density derivatives of the energy 
%at saturation density. 
%The parametrization of the surface tension entering the 
%expression of the surface plus curvature energy was suggested from TF 
%calculations at extreme isospin ratios~\cite{Ravenhall1983}, and the associated 
%parameters are fitted to experimental masses.
%The fact that our CLDM is based on the metamodel allows to calculate the 
%cluster energy for any functional of nuclear matter with a unique functional 
%form.
%%
%The cluster energy enters the system of four coupled differential equations 
%obtained from the minimization of the WS cell energy density at fixed baryon 
%density, which once solved numerically, gives the ground-state composition 
%and so the EoS of the inner crust.
%%
%Using the empirical parameters associated to the recent and realistic BSk24 
%functional~\cite{Goriely2013}, we have observed a very good agreement with more 
%microscopic approaches, notably ETF calculations, in particular for the value 
%of the equilibrium cluster proton number, $Z\approx 40$, throughout the inner 
%crust. We have seen that the EoS at subnuclear densities is correlated 
%with the symmetry energy at saturation density $E_{sym}$, which is the most
%constrained empirical parameter in the isovector channel. 
%%
%The main drawback of the CLD approach is that quantum effects are 
%lost. While in the free neutron regime neutron shell effects become 
%vanishingly small~\cite{Chamel2006,Chamel2007}, proton shell effects are 
%expected to persist at zero temperature. 
%For recent BSk functionals, proton shell corrections in the free neutron regime 
%were calculated using Strutinsky method~\cite{Pearson2018}. We have added 
%these corrections on top of the corresponding CLD energy to recover the 
%appearance of the magic numbers in the ground-state composition. 
%A remarkable stability has been
%observed at $Z=40$ for BSk24 and BSk26, as well as $Z=40$ and $Z=20$ for BSk22,
%in a very good agreement with the ETFSI results of~\cite{Pearson2018}.
%%
%One of the virtues of the CLD approach is that different geometries for 
%clusters can be considered simply by adding a dimensionality parameter to the 
%surface and Coulomb terms, allowing for the study of nonspherical pasta phases 
%potentially present in the deepest layers of the inner crust 
%\cite{Ravenhall1983,Lattimer1991,Lorenz1993,Newton2012}. 
%For each of the four models considered, we have obtained the sequence spheres 
%$\longrightarrow$ cylinders $\longrightarrow$ plates $\longrightarrow$ tubes, 
%with spheres dominating up to $n_B \approx 0.05$ fm$^{-3}$. The sequence of
%pasta phases is expected to be sensitive to the behavior or the surface 
%tension at extreme isospin ratios, which is poorly constrained to the present
%day.
%%
%The location of the transition point to homogeneous $npe$ matter has been
%computed from the crust side, in contrast with the so-called
%thermodynamical~\cite{Gonzalez2017} and 
%dynamical~\cite{Pethick1995cc,Antic2019} methods based on the spinodal 
%decomposition scenario, which is not compatible with clusterized matter.
%% We have obtained a good agreement with the dynamical method. We have
%% observed apparent correlations with $L_{sym}$ and with the parameter
%% governing the behavior of the surface tension at high isospin.

\clearpage\thispagestyle{empty}
