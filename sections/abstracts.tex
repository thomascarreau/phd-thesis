\clearpage
\thispagestyle{empty} %empty
\newpage
\enlargethispage{2\baselineskip}

\vspace*{-6\baselineskip}
\begin{changemargin}{-0.2cm}{-1.2cm} 
\noindent
\begin{center}
\textbf{Modeling the (proto)neutron star crust: toward a controlled estimation
of uncertainties}
\end{center}

% PROBLEM
The main aim of this thesis is to make realistic predictions and to
investigate the sources of uncertainties in the observables of nonaccreting
cold neutron stars and warm protoneutron stars, using the present day 
constraints provided by nuclear experiments, developments in chiral 
effective field theory, and astrophysical observations.
% METHODS
A unified metamodeling approach was introduced to calculate the stellar 
composition and equation of state of cold nonaccreting neutron stars for any 
functional of nuclear matter.
A Bayesian determination of the equation of state parameters was carried out,
leading to realistic predictions for neutron star observables.
At finite temperature, a full statistical equilibrium of ions in the crust was 
considered, allowing in particular for the computation of the impurity 
parameter.
% RESULTS
The results are compatible with constraints inferred from GW170817, and
suggest that a full crustal origin of pulsar glitches should be excluded. 
Deviations in the crust composition from cold catalyzed matter are 
observed at the crystallization temperature. 
Results show that the contribution of impurities is nonnegligible, 
thus potentially having an impact on transport properties in the crust.
% CONCLUSION
Higher precision in the determination of high-order isovector
empirical parameters through nuclear experiments or low-density effective field 
theory predictions, and in the experimental and/or theoretical knowledge of the 
surface energy at extreme isospin ratios are needed to reduce the uncertainties 
of crustal observables. 
The numerical framework developed during this thesis can be used as a basis for 
future studies.

\noindent
\textbf{Keywords:} 
neutron star, crust, nuclear physics, equation of state, Bayesian statistics, 
pulsar glitch, nuclear statistical equilibrium, crystallization 

\begin{center}
  \noindent\rule[0.5ex]{0.8\linewidth}{1pt}
\end{center}

\begin{center}
  \textbf{Modélisation de la croûte des (proto)étoiles à neutrons~: vers une
  estimation contrôlée des incertitudes}
\end{center}

% PROBLEM
L'objectif principal de cette thèse est de fournir des prédictions réalistes et 
d'étudier les sources d'incertitudes sur les observables des étoiles à 
neutrons froides isolées et des protoétoiles à neutrons chaudes. Ces
prédictions doivent s'accorder avec les contraintes actuelles fournies par 
les expériences en physique nucléaire, les développements en théorie des perturbations chirales et les observations astrophysiques.
% METHODS
Une approche unifiée de métamodélisation a été introduite pour calculer la 
composition et l'équation d'état des étoiles à neutrons froides isolées pour 
toute fonctionnelle de matière nucléaire.
Une détermination bayésienne des paramètres de l'équation d'état a été 
effectuée, conduisant à des prédictions réalistes pour les observables
de l'étoile.
À température finie, l'équilibre statistique complet des ions dans la croûte 
a été considéré, permettant entre autres le calcul du paramètre d'impureté.
% RESULTS
Les résultats sont compatibles avec les contraintes déduites de GW170817 et 
tendent à indiquer que le phénomène de glitch ne peut pas uniquement tirer son 
origine de la physique de la croûte.
Des déviations dans la composition de la croûte par rapport à la matière 
catalysée froide sont observées à la 
température de cristallisation. Les résultats montrent que la contribution des 
impuretés n'est pas négligeable, ce qui pourrait avoir un impact sur les 
propriétés de transport dans la croûte.
% CONCLUSION
Une plus grande précision dans la détermination des dérivées 
d'ordre élevé de l'énergie de symétrie dans les expériences et/ou la théorie 
des perturbations chirales, et dans la connaissance de l'énergie de surface à 
des isospins extrêmes est nécessaire pour réduire les incertitudes sur les 
observables de la croûte. 
Le cadre numérique développé au cours de cette thèse peut servir de 
base à des recherches futures.

\textbf{Mots clés~:} 
étoile à neutrons, croûte, physique nucléaire, équation d'état, statistiques 
bayésiennes, glitch de pulsar, équilibre statistique nucléaire, cristallisation

\end{changemargin}

%\clearpage
\thispagestyle{empty}
