\appendix
\chapter{Energy density of a relativistic electron gas}\label{appendix:epse}

We give here the derivation of the energy density of a relativistic electron
gas of density $n_e$ at zero temperature.

From $\approx 10^{14}$ g/cm$^3$, electrons are essentially free. In this
case, the energy density is given by
%
\begin{equation}
  \varepsilon_e = \frac{c}{\pi^2}\int_{k=0}^{k_{e}} dk k^2 
  \sqrt{\hbar^2 k^2 + m_e^2c^2},
\end{equation}
%
$c$ being the speed of light, $\hbar$ the reduced Planck constant, and $m_e$
the electron mass. The electron Fermi wave number $k_e$ is related to the
electron density via $k_e = (3\pi^2n_e)^{1/3}$. Making a simple variable, 
$x = \hbar k/(m_ec)$, we can rewrite the energy density as
%
\begin{equation}
  \varepsilon_e = \frac{m_e^4c^5}{\pi^2\hbar^3}\xi(x_r)
\end{equation}
%
with $x_r = \hbar k_e/(m_ec)$ and
%
\begin{equation}
  \xi = \int_{x=0}^{x_r} dx x^2\sqrt{x^2+1}.
\end{equation}
%
Integrating by parts, it follows
%
\begin{eqnarray}
  \xi &=& \left[\frac{x^3}{3}\sqrt{x^2 + 1}\right]_0^{x_r} -
  \int_0^{x_r}dx\frac{x^4}{3\sqrt{x^2+1}}\notag\\
  3\xi &=& x_r^3\sqrt{x_r^2+1} 
  - \int_0^{x_r}dx x^2 \left(\frac{x^2+1}{\sqrt{x^2
  + 1}} - \frac{1}{\sqrt{x^2+1}}\right)\notag\\
        4\xi &=& x_r^3\sqrt{x_r^2+1} + \int_0^{x_r}dx \frac{x^2}{\sqrt{x^2+1}}.
\end{eqnarray}
%
We have
%
\begin{equation}
  \frac{d}{dx}\left(x\sqrt{x^2 + 1}\right) = \sqrt{x^2+1} 
  + \frac{x^2}{\sqrt{x^2+1}},
\end{equation}
%
and
%
\begin{equation}
  \frac{d}{dx}\left[\ln(x + \sqrt{x^2+1})\right] = \sqrt{x^2+1} -
  \frac{x^2}{\sqrt{x^2+1}}.
\end{equation}
%
thus,
%
\begin{equation}
  \frac{1}{2}\frac{d}{dx}\left[x\sqrt{x^2+1} -
  \ln\left(x+\sqrt{x^2+1}\right)\right] = \frac{x^2}{\sqrt{x^2+1}}.
\end{equation}
%
Hence,
%
\begin{equation}
  4\xi = x_r^3\sqrt{x_r^2+1} + \frac{1}{2}\left[x\sqrt{x^2+1} -
  \ln\left(x+\sqrt{x^2+1}\right)\right]_0^{x_r},
\end{equation}
%
which finally gives
%
\begin{equation}
  \varepsilon_e(n_e) = \frac{m_e^4c^5}{8\pi^2\hbar^3}\left[x_r(2x_r^2 +
  1)\gamma_r - \ln(x_r + \gamma_r)\right], 
\end{equation}
%
where we have introduced the parameter parameter $\gamma_r = \sqrt{x_r^2+1}$.

\chapter{Neutron and proton chemical potentials in the 
metamodel}\label{appendix:mutau}

From the metamodeling of the nuclear matter energy, Eq.~\ref{eq:meta}, we give 
here the complete expressions of the neutron and proton chemical potentials
for a homogeneous system characterized by a neutron density $n_n$ and proton
density $n_p$ at zero temperature. Let us introduce the variables 
$x=(n-n_{sat})/(3n_{sat})$ and $\delta = (n_n - n_p)/n$, $n=n_n+n_p$ being the
total baryon density.

The neutron chemical potential, with the rest mass energy $m_nc^2$, is given by
%
\begin{eqnarray}
  \mu_{HM,n} &=& e_{HM}(n_n,n_p) + n\left(\frac{\partial
  e_{HM}}{\partial n_n}\right)_{n_p} + m_nc^2\\
  \mu_{HM,n} &=& e_{HM}(n,\delta) + \frac{1+3x}{3}\left(\frac{\partial
  e_{HM}}{\partial x}\right)_\delta + (1 - \delta)\left(\frac{\partial
e_{HM}}{\partial \delta}\right)_x + m_n c^2.
\end{eqnarray}
%
Equivalently, the proton chemical potential is given by
%
\begin{equation}
  \mu_{HM,p} = e_{HM}(n,\delta) + \frac{1+3x}{3}\left(\frac{\partial
  e_{HM}}{\partial x}\right)_\delta - (1 + \delta)\left(\frac{\partial
e_{HM}}{\partial \delta}\right)_x + m_p c^2.
\end{equation}
%
We first go through the derivative with respect to $x$, yielding
%
\begin{eqnarray}
  \left(\frac{\partial e_{HM}}{\partial x}\right)_\delta &=&  
  \frac{5}{1+3x}t_{HM}^{FG}(n,\delta) \notag\\
                                                         &&-
  \frac{3}{1+3x}\frac{1}{2}t_{sat}^{FG}(1+3x)^{2/3}
  \left[(1+\delta)^{5/3}+(1-\delta)^{5/3}\right]\notag\\
                                                         &&+
  \sum_{\alpha \geq 0}
  (v_\alpha^{is} + v_\alpha^{iv}\delta^2)\frac{1}{\alpha!}\left(\alpha
  x^{\alpha-1}u_\alpha(x) + x^\alpha \frac{du_\alpha}{dx}\right).
\end{eqnarray}
%
Let us notice that is gives the expression for the total pressure,
%
\begin{eqnarray}
  P_{HM}(n,\delta) &=& n^2\left(\frac{\partial e_{HM}}{\partial
  n}\right)_\delta\\
  &=& \frac{1}{3}n_{sat}(1+3x)^2\left(\frac{\partial e_{HM}}{\partial
  x}\right)_\delta.
\end{eqnarray}
%
We now turn to the derivative with respect to the asymmetry $\delta$,
%
\begin{eqnarray}
  \left(\frac{\partial e_{HM}}{\partial \delta}\right)_x &=&
  \frac{1}{2}t_{sat}^{FG}(1+3x)^{5/3}\left[(1+\delta)^{5/3} 
  - (1-\delta)^{5/3}\right]\notag\\
                                                         &&+ 
  \frac{5}{6}t_{sat}^{FG}(1+3x)^{2/3}\left[(1+\delta)^{2/3}\frac{m}{m_n^*} -
  (1-\delta)^{2/3}\frac{m}{m_p^*}\right]\notag\\
                                                         &&+
  2\delta \sum_{\alpha \geq 0}v_\alpha^{iv}\frac{x^\alpha}{\alpha!}u_\alpha(x).
\end{eqnarray}
%

\chapter{R\'esum\'e en fran\c{c}ais}\label{appendix:fr}
