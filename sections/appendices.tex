\appendix
\chapter{Energy density of a relativistic electron gas}\label{appendix:epse}

We give here the derivation of the energy density of a relativistic electron
gas of density $n_e$ at zero temperature.

From $\approx 10^{14}$ g/cm$^3$, electrons are essentially free. In this
case, the energy density is given by
%
\begin{equation}
  \varepsilon_e = \frac{c}{\pi^2}\int_{k=0}^{k_{e}} dk k^2 
  \sqrt{\hbar^2 k^2 + m_e^2c^2},
\end{equation}
%
$c$ being the speed of light, $\hbar$ the reduced Planck constant, and $m_e$
the electron mass. The electron Fermi wave number $k_e$ is related to the
electron density via $k_e = (3\pi^2n_e)^{1/3}$. Making a simple variable, 
$x = \hbar k/(m_ec)$, we can rewrite the energy density as
%
\begin{equation}
  \varepsilon_e = \frac{m_e^4c^5}{\pi^2\hbar^3}\xi(x_r)
\end{equation}
%
with $x_r = \hbar k_e/(m_ec)$ and
%
\begin{equation}
  \xi = \int_{x=0}^{x_r} dx x^2\sqrt{x^2+1}.
\end{equation}
%
Integrating by parts, it follows
%
\begin{eqnarray}
  \xi &=& \left[\frac{x^3}{3}\sqrt{x^2 + 1}\right]_0^{x_r} -
  \int_0^{x_r}dx\frac{x^4}{3\sqrt{x^2+1}}\notag\\
  3\xi &=& x_r^3\sqrt{x_r^2+1} 
  - \int_0^{x_r}dx x^2 \left(\frac{x^2+1}{\sqrt{x^2
  + 1}} - \frac{1}{\sqrt{x^2+1}}\right)\notag\\
        4\xi &=& x_r^3\sqrt{x_r^2+1} + \int_0^{x_r}dx \frac{x^2}{\sqrt{x^2+1}}.
\end{eqnarray}
%
We have
%
\begin{equation}
  \frac{d}{dx}\left(x\sqrt{x^2 + 1}\right) = \sqrt{x^2+1} 
  + \frac{x^2}{\sqrt{x^2+1}},
\end{equation}
%
and
%
\begin{equation}
  \frac{d}{dx}\left[\ln(x + \sqrt{x^2+1})\right] = \sqrt{x^2+1} -
  \frac{x^2}{\sqrt{x^2+1}}.
\end{equation}
%
thus,
%
\begin{equation}
  \frac{1}{2}\frac{d}{dx}\left[x\sqrt{x^2+1} -
  \ln\left(x+\sqrt{x^2+1}\right)\right] = \frac{x^2}{\sqrt{x^2+1}}.
\end{equation}
%
Hence,
%
\begin{equation}
  4\xi = x_r^3\sqrt{x_r^2+1} + \frac{1}{2}\left[x\sqrt{x^2+1} -
  \ln\left(x+\sqrt{x^2+1}\right)\right]_0^{x_r},
\end{equation}
%
which finally gives
%
\begin{equation}
  \varepsilon_e(n_e) = \frac{m_e^4c^5}{8\pi^2\hbar^3}\left[x_r(2x_r^2 +
  1)\gamma_r - \ln(x_r + \gamma_r)\right], 
\end{equation}
%
where we have introduced the parameter parameter $\gamma_r = \sqrt{x_r^2+1}$.

\chapter{Neutron and proton chemical potentials in the 
metamodel}\label{appendix:mutau}

From the metamodeling of the nuclear matter energy, Eq.~\ref{eq:meta}, we give 
here the complete expressions of the neutron and proton chemical potentials
for a homogeneous system characterized by a neutron density $n_n$ and proton
density $n_p$ at zero temperature. Let us introduce the variables 
$x=(n-n_{sat})/(3n_{sat})$ and $\delta = (n_n - n_p)/n$, $n=n_n+n_p$ being the
total baryon density.

The neutron chemical potential, with the rest mass energy $m_nc^2$, is given by
%
\begin{eqnarray}
  \mu_{HM,n} &=& e_{HM}(n_n,n_p) + n\left(\frac{\partial
  e_{HM}}{\partial n_n}\right)_{n_p} + m_nc^2\\
  \mu_{HM,n} &=& e_{HM}(n,\delta) + \frac{1+3x}{3}\left(\frac{\partial
  e_{HM}}{\partial x}\right)_\delta + (1 - \delta)\left(\frac{\partial
e_{HM}}{\partial \delta}\right)_x + m_n c^2.
\end{eqnarray}
%
Equivalently, the proton chemical potential is given by
%
\begin{equation}
  \mu_{HM,p} = e_{HM}(n,\delta) + \frac{1+3x}{3}\left(\frac{\partial
  e_{HM}}{\partial x}\right)_\delta - (1 + \delta)\left(\frac{\partial
e_{HM}}{\partial \delta}\right)_x + m_p c^2.
\end{equation}
%
We first go through the derivative with respect to $x$, yielding
%
\begin{eqnarray}
  \left(\frac{\partial e_{HM}}{\partial x}\right)_\delta &=&  
  \frac{5}{1+3x}t_{HM}^{FG}(n,\delta) \notag\\
                                                         &&-
  \frac{3}{1+3x}\frac{1}{2}t_{sat}^{FG}(1+3x)^{2/3}
  \left[(1+\delta)^{5/3}+(1-\delta)^{5/3}\right]\notag\\
                                                         &&+
  \sum_{\alpha \geq 0}
  (v_\alpha^{is} + v_\alpha^{iv}\delta^2)\frac{1}{\alpha!}\left(\alpha
  x^{\alpha-1}u_\alpha(x) + x^\alpha \frac{du_\alpha}{dx}\right).
\end{eqnarray}
%
Let us notice that is gives the expression for the total pressure,
%
\begin{eqnarray}
  P_{HM}(n,\delta) &=& n^2\left(\frac{\partial e_{HM}}{\partial
  n}\right)_\delta\\
  &=& \frac{1}{3}n_{sat}(1+3x)^2\left(\frac{\partial e_{HM}}{\partial
  x}\right)_\delta.
\end{eqnarray}
%
We now turn to the derivative with respect to the asymmetry $\delta$,
%
\begin{eqnarray}
  \left(\frac{\partial e_{HM}}{\partial \delta}\right)_x &=&
  \frac{1}{2}t_{sat}^{FG}(1+3x)^{5/3}\left[(1+\delta)^{5/3} 
  - (1-\delta)^{5/3}\right]\notag\\
                                                         &&+ 
  \frac{5}{6}t_{sat}^{FG}(1+3x)^{2/3}\left[(1+\delta)^{2/3}\frac{m}{m_n^*} -
  (1-\delta)^{2/3}\frac{m}{m_p^*}\right]\notag\\
                                                         &&+
  2\delta \sum_{\alpha \geq 0}v_\alpha^{iv}\frac{x^\alpha}{\alpha!}u_\alpha(x).
\end{eqnarray}
%

\chapter{Résumé en français}\label{appendix:fr_part}

\section{Conclusions générales et perspectives}

L'objectif principal de cette thèse a été de fournir des prédictions réalistes 
et d'étudier les sources d'incertitudes sur les observables des étoiles à 
neutrons froides isolées et des protoétoiles à neutrons chaudes, en utilisant 
une métamodélisation unifiée pour décrire l'énergétique de la matière d'étoile, 
qui permet de tenir compte des contraintes actuelles liées aux expériences 
nucléaires et des observations astrophysiques.
\\

Nous avons envisagé une approche de métamodélisation unifiée afin de 
calculer la composition et l'équation d'état de l'étoile à neutrons froide
isolée pour toute fonctionnelle de matière nucléaire en utilisant les 
paramètres empiriques associés comme seuls ingrédients. Nous avons obtenu une
erreur qui, dans le cas de l'équation d'état basée sur SLy4, est inférieure à 
$1\%$. 
Pour modéliser l'énergie des clusters dans la croûte interne, nous avons 
proposé une version du célèbre modèle de la goutte liquide compressible basée 
sur la technique de métamodélisation, avec un paramétrage de la tension de 
surface suggéré par des calculs Thomas-Fermi à des ratios d'isospins extrêmes. 
%
La composition obtenue dans l'état fondamental suit de près les résultats de 
calculs Thomas-Fermi étendu rapportés dans la littérature. Le principal 
inconvénient de l'approche de la goutte liquide compressible est que les effets 
quantiques sont perdus. Cependant, nous avons montré que les nombres magiques 
dans la croûte interne peuvent être retrouvés en ajoutant les corrections de 
couches calculées par la méthode de Strutinsky, conduisant alors à un très bon 
accord avec les résultats de calculs Thomas-Fermi étendu avec corrections de
couches Strutinsky. 
%
La même séquence de phases non sphériques dans les couches les plus profondes 
de la croûte interne a été observée pour tous les modèles considérés, mais nous 
avons souligné qu'elle est sensible au comportement de la tension de surface à 
grand isospin, qui reste peu contrainte à ce jour. 
%
L'incertitude sur l'équation d'état induite par le traitement de l'énergie de 
surface peut être estimé à $10\%$, ce qui correspond à la différence entre 
nos résultats avec la fonctionnelle SLy4 et l'équation d'état de Douchin et
Haensel qui utilise la même fonctionnelle mais un traitement différent de 
l'énergie de surface. 
Il en va de même pour l'emplacement du point de transition vers 
la matière homogène $npe$, qui a été calculée pour plusieurs modèles nucléaires
depuis la croûte. 
%
Nous avons montré que l'anticorrélation entre la pente de l'énergie de symétrie
à la densité de saturation $L_{sym}$ et la densité de transition croûte-coeur, 
signalée dans de nombreux travaux antérieurs, est obtenue si l'énergie de 
surface est seulement optimisée sur un ensemble limité de noyaux magiques et 
semi-magiques sphériques. Inversement, si la tension de surface est optimisée 
sur un grand nombre de données et si des termes de courbure sont ajoutés,
cette corrélation disparaît et la sensibilité à l'équation d'état est limitée
aux paramètres empiriques d'ordres supérieurs dans le secteur isovecteur
($K_{sym}$, $Q_{sym}$), qui restent peu contraints à ce jour.

Nous avons exploité le principal atout de la technique de métamodélisation, à 
savoir le fait qu'aucune corrélation artificielle n'est introduite a priori 
entre les paramètres empiriques, pour effectuer la détermination bayésienne ces
derniers, conduisant à des prédictions réalistes pour les observables des 
étoile à neutrons. Nous avons considéré a priori une distribution uniforme pour 
les paramètres empiriques dont les limites sont compatible avec les 
contraintes expérimentales. 
%
La fonction de vraisemblance que nous avons construite tient compte de 
récents résultats de la théorie des perturbations chirales pour la matière 
nucléaire symétrique et la matière pure de neutrons jusqu'à 0.20 fm$^{-3}$,
ainsi que de la contrainte de masse maximale de l'étoile et des principes
physiques de base. 
Elle comprend également une probabilité dans laquelle est encodée la capacité 
du modèle de goutte liquide compressible à reproduire les masses de l'AME2016. 
Nous avons montré que le fait d'imposer les contraintes 
associées aux calculs ab initio est très efficace pour contraindre les 
paramètres empiriques dans le secteur isovecteur, et que cela donne lieu à des 
corrélations entre les dérivés de l'énergie de symétrie à la densité de 
saturation.
Nous avons fait des prédictions générales pour les propriétés statiques en 
utilisant la distribution posterior des paramètres empiriques et avons constaté 
que nos résultats sont compatibles avec les contraintes des collaborations LIGO 
et Virgo déduites de l'évènement GW170817.
%
La seule hypothèse de la technique de métamodélisation étant la 
possibilité d'étendre l'équation d'état en une série de Taylor, ce résultat 
implique que nous n'avons pas de preuve irréfutable d'une transition de la 
phase de premier ordre dans le coeur des étoiles à neutrons, bien que cela
ne puisse évidemment pas être exclu.
%
Le moment d'inertie de la croûte, qui est fortement corrélé avec la 
localisation du point de transition croûte-coeur, a été calculé, et les
résultats tendent à indiquer que le phénomène de glitch ne peut pas uniquement
tirer son origine de la physique de la croûte. 
Cela ouvre des possibilités intéressantes concernant la pertinence des 
composantes superfluides dans le coeur, en particulier dans le canal $^3P_2$ 
pour les paires $nn$ et $pp$.

Dans la continuité d'un travail récent sur la croûte externe, nous avons 
envisagé un équilibre statistique complet des ions dans la croûte à température 
finie permettant la présence d'un gaz de neutrons. Nous avons évalué 
l'abondance des noyaux de masse impaire et des noyaux de charge impaire 
présents dans la croûte externe à la température de cristallisation. 
Leur présence est intéressante, car elle pourrait causer des transitions de
phase ferromagnétiques.
%
Nous avons considéré une dépendance en température des corrections de couches 
des protons dans le régime de la croûte interne. La température de 
cristallisation et la composition associée ont été calculées dans 
l'approximation de plasma à un composant, et nos résultats suggèrent que la 
source la plus importante de dépendance au modèle provient de la partie lisse 
de la fonctionnelle, l'ingrédient le plus important à fixer pour une la 
prédiction quantitative des propriétés de la croûte interne étant la tension de 
superface à des ratios d'isospins extrêmes. 
%
Des écarts par rapport à la matière froide catalysée ont été observés à faible 
densité et température de cristallisation, ce qui pourrait avoir un impact sur 
les simulations du processus $r$.
Enfin, nous avons calculé le paramètre d'impureté de manière systématique dans
la croûte interne au point de cristallisation pour quatre fonctionnelles BSk
récentes qui tiennent compte des incertitudes actuelles sur l'énergie de 
symétrie. 
Il s'agit à ce jour du premier calcul du facteur d'impureté dans la croûte
interne basé sur des fonctionnelles nucléaires réalistes, et il a montré 
que la contribution des impuretés est non négligeable, ce qui pourrait modifier 
les propriétés de transport dans la croûte.
\\

Il est clair que la détermination expérimentale des paramètres empiriques
d'ordres supérieures du secteur isovecteur dans les prévisions théoriques de 
la théorie des perturbations chirales à basse densité et dans la modélisation
microscopique de l'énergie de surface à des ratios d'isospins extrêmes 
sont nécessaires pour réduire les incertitudes des observables de la croûte des
étoiles à neutrons. 
%
De plus, nous avons montré que les contraintes sur les propriétés de la matière 
dense peuvent être déduites des observations astrophysiques dans un cadre 
bayésien. 
De nombreuses nouvelles mesures sont attendues dans un futur proche de la part
de NICER et de LIGO/Virgo. 
Nous pouvons donc espérer réduire les incertitudes sur les dérivés de l'énergie 
de symétrie d'ordres élevés et sur les observables de l'étoile en utilisant une 
méthode bayésienne ou l'apprentissage machine.

Diverses applications du formalisme introduit pour la description d'un plasma à
plusieurs composantes dans un équilibre statistique complet peut être envisagé 
à l'avenir. 
Par exemple, il est envisageable d'étudier la présence d'hyperons dans la 
croûte au point de cristallisation. 
Les géométries non sphériques peuvent également être prises en compte dans ce
traitement, ce qui permettrait l'évaluation du paramètre d'impureté dans les 
couches profondes de la croûte, pouvant ainsi vérifier l'hypothèse 
de~\cite{Pons2013} que la présence d'une couche hautement résistive dans la 
croûte interne pourrait conduire à une limite plus élevée de la période de 
rotation des pulsars X.

Durant cette thèse, j'ai écrit une bibliothèque open source en langage C,
NSEoS, dans le but de fournir des outils utiles liés à la physique des étoiles 
à neutrons~\cite{NSEoS}. La bibliothèque a été utilisée pour produire tous les 
résultats présentés dans cette thèse.
De plus, elle a déjà été utilisée par différents étudiants pendant leur stage 
sous la supervision conjointe de mon directeur de thèse et de moi-même, et 
nous pensons qu'elle peut servir de base à de futures études.
