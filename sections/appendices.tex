\appendix
\chapter{Energy density of a relativistic electron gas}\label{appendix:epse}

We give here the derivation of the energy density of a relativistic electron
gas of density $n_e$ at zero temperature.

From $\approx 10^{14}$ g/cm$^3$, electrons are essentially free. In this
case, the energy density is given by
%
\begin{equation}
  \varepsilon_e = \frac{c}{\pi^2}\int_{k=0}^{k_{e}} dk k^2 
  \sqrt{\hbar^2 k^2 + m_e^2c^2},
\end{equation}
%
$c$ being the speed of light, $\hbar$ the reduced Planck constant, and $m_e$
the electron mass. The electron Fermi wave number $k_e$ is related to the
electron density via $k_e = (3\pi^2n_e)^{1/3}$. Making a simple variable, 
$x = \hbar k/(m_ec)$, we can rewrite the energy density as
%
\begin{equation}
  \varepsilon_e = \frac{m_e^4c^5}{\pi^2\hbar^3}\xi(x_r)
\end{equation}
%
with $x_r = \hbar k_e/(m_ec)$ and
%
\begin{equation}
  \xi = \int_{x=0}^{x_r} dx x^2\sqrt{x^2+1}.
\end{equation}
%
Integrating by parts, it follows
%
\begin{eqnarray}
  \xi &=& \left[\frac{x^3}{3}\sqrt{x^2 + 1}\right]_0^{x_r} -
  \int_0^{x_r}dx\frac{x^4}{3\sqrt{x^2+1}}\notag\\
  3\xi &=& x_r^3\sqrt{x_r^2+1} 
  - \int_0^{x_r}dx x^2 \left(\frac{x^2+1}{\sqrt{x^2
  + 1}} - \frac{1}{\sqrt{x^2+1}}\right)\notag\\
        4\xi &=& x_r^3\sqrt{x_r^2+1} + \int_0^{x_r}dx \frac{x^2}{\sqrt{x^2+1}}.
\end{eqnarray}
%
We have
%
\begin{equation}
  \frac{d}{dx}\left(x\sqrt{x^2 + 1}\right) = \sqrt{x^2+1} 
  + \frac{x^2}{\sqrt{x^2+1}},
\end{equation}
%
and
%
\begin{equation}
  \frac{d}{dx}\left[\ln(x + \sqrt{x^2+1})\right] = \sqrt{x^2+1} -
  \frac{x^2}{\sqrt{x^2+1}}.
\end{equation}
%
thus,
%
\begin{equation}
  \frac{1}{2}\frac{d}{dx}\left[x\sqrt{x^2+1} -
  \ln\left(x+\sqrt{x^2+1}\right)\right] = \frac{x^2}{\sqrt{x^2+1}}.
\end{equation}
%
Hence,
%
\begin{equation}
  4\xi = x_r^3\sqrt{x_r^2+1} + \frac{1}{2}\left[x\sqrt{x^2+1} -
  \ln\left(x+\sqrt{x^2+1}\right)\right]_0^{x_r},
\end{equation}
%
which finally gives
%
\begin{equation}
  \varepsilon_e(n_e) = \frac{m_e^4c^5}{8\pi^2\hbar^3}\left[x_r(2x_r^2 +
  1)\gamma_r - \ln(x_r + \gamma_r)\right], 
\end{equation}
%
where we have introduced the parameter parameter $\gamma_r = \sqrt{x_r^2+1}$.

\chapter{Neutron and proton chemical potentials in the 
metamodel}\label{appendix:mutau}

From the metamodeling of the nuclear matter energy, Eq.~\ref{eq:meta}, we give 
here the complete expressions of the neutron and proton chemical potentials
for a homogeneous system characterized by a neutron density $n_n$ and proton
density $n_p$ at zero temperature. Let us introduce the variables 
$x=(n-n_{sat})/(3n_{sat})$ and $\delta = (n_n - n_p)/n$, $n=n_n+n_p$ being the
total baryon density.

The neutron chemical potential, with the rest mass energy $m_nc^2$, is given by
%
\begin{eqnarray}
  \mu_{HM,n} &=& e_{HM}(n_n,n_p) + n\left(\frac{\partial
  e_{HM}}{\partial n_n}\right)_{n_p} + m_nc^2\\
  \mu_{HM,n} &=& e_{HM}(n,\delta) + \frac{1+3x}{3}\left(\frac{\partial
  e_{HM}}{\partial x}\right)_\delta + (1 - \delta)\left(\frac{\partial
e_{HM}}{\partial \delta}\right)_x + m_n c^2.
\end{eqnarray}
%
Equivalently, the proton chemical potential is given by
%
\begin{equation}
  \mu_{HM,p} = e_{HM}(n,\delta) + \frac{1+3x}{3}\left(\frac{\partial
  e_{HM}}{\partial x}\right)_\delta - (1 + \delta)\left(\frac{\partial
e_{HM}}{\partial \delta}\right)_x + m_p c^2.
\end{equation}
%
We first go through the derivative with respect to $x$, yielding
%
\begin{eqnarray}
  \left(\frac{\partial e_{HM}}{\partial x}\right)_\delta &=&  
  \frac{5}{1+3x}t_{HM}^{FG}(n,\delta) \notag\\
                                                         &&-
  \frac{3}{1+3x}\frac{1}{2}t_{sat}^{FG}(1+3x)^{2/3}
  \left[(1+\delta)^{5/3}+(1-\delta)^{5/3}\right]\notag\\
                                                         &&+
  \sum_{\alpha \geq 0}
  (v_\alpha^{is} + v_\alpha^{iv}\delta^2)\frac{1}{\alpha!}\left(\alpha
  x^{\alpha-1}u_\alpha(x) + x^\alpha \frac{du_\alpha}{dx}\right).
\end{eqnarray}
%
Let us notice that is gives the expression for the total pressure,
%
\begin{eqnarray}
  P_{HM}(n,\delta) &=& n^2\left(\frac{\partial e_{HM}}{\partial
  n}\right)_\delta\\
  &=& \frac{1}{3}n_{sat}(1+3x)^2\left(\frac{\partial e_{HM}}{\partial
  x}\right)_\delta.
\end{eqnarray}
%
We now turn to the derivative with respect to the asymmetry $\delta$,
%
\begin{eqnarray}
  \left(\frac{\partial e_{HM}}{\partial \delta}\right)_x &=&
  \frac{1}{2}t_{sat}^{FG}(1+3x)^{5/3}\left[(1+\delta)^{5/3} 
  - (1-\delta)^{5/3}\right]\notag\\
                                                         &&+ 
  \frac{5}{6}t_{sat}^{FG}(1+3x)^{2/3}\left[(1+\delta)^{2/3}\frac{m}{m_n^*} -
  (1-\delta)^{2/3}\frac{m}{m_p^*}\right]\notag\\
                                                         &&+
  2\delta \sum_{\alpha \geq 0}v_\alpha^{iv}\frac{x^\alpha}{\alpha!}u_\alpha(x).
\end{eqnarray}
%

\chapter{Résumé en français}\label{appendix:fr_part}

\section{Chapitre 1}

Dans ce premier chapitre, nous avons d'abord évalué l'état fondamental de la 
croûte externe, qui est obtenu par application de la méthode proposée par Baym,
Pethick et Sutherland en 1971~\cite{BPS}, en utilisant les plus récentes
données sur les masses \cite{Huang2017,Welker2017} complétées par des 
des tables de masses théoriques Hartree-Fock-Bogoliubov~\cite{Samyn2002},
qui sont basées sur la théorie de la fonctionnelle de la densité. 
Nous avons montré que la composition de la croûte externe et donc l'équation
d'état sont entièrement déterminées par les connaissances actuelles sur les 
masses expérimentales jusqu'à $n_B \approx 3\times 10^{-5}$ fm$^{-3}$.
À des densités plus élevées, nous avons observé la persistance de $N=82$ pour 
chacun des quatre modèles de masse considérés.

Nous avons proposé une version du modèle de la goutte liquide compressible
basée sur la technique de métamodélisation 
\cite{Margueron2018a,Margueron2018b}. Nous avons vu que le métamodèle offre la 
possibilité de reproduire toute fonctionnelle de matière nucléaire très 
précisément, avec une forme fonctionnelle unique. La paramétrisation de 
la tension de surface a été suggérée par des calculs microscopiques dans le 
régime des neutrons libres~\cite{Ravenhall1983} et les paramètres de surface et 
de courbure sont ajustés sur les masses expérimentales. Nous avons utilisé 
notre modèle de la goutte liquide compressible pour calculer l'état fondamental 
de la croûte interne, qui est obtenu en minimisant la densité d'énergie de la 
matière à densité baryonique constante sous la condition de neutralité 
de la charge globale. Nous avons ainsi abouti à un système de quatre équations 
différentielles couplées, correspondant à des conditions d'équilibre, que nous 
avons résolues numériquement en utilisant la méthode de Broyden.
%
La composition de la croûte interne dans l'état fondamental pour les paramètres 
empiriques du BSk24 a été présenté et un très bon accord avec des approches 
plus microscopiques a été observée concernant la valeur de $Z \approx 40$. Nous 
avons également vu que la fraction de protons diminue avec la densité de façon 
monotone, comme dans la croûte externe. 
L'équation d'état de la croûte interne a été calculée et une corrélation 
positive avec le paramètre empirique $E_{sym}$ a été révélée. Nous avons montré 
que les nombres magiques, qui disparaissent dans le cadre de l'approche
classique de la goutte liquide compressible, peuvent être retrouvés en ajoutant 
de manière perturbative des corrections de couches des protons calculées par la 
méthode de Strutinsky. De cette façon, un très bon accord avec les calculs
Thomas-Fermi étendu avec corrections de couches Strutinksy a été observé pour 
les functionelles BSk récentes~\cite{Pearson2018}. Avec notre modèle de la
goutte liquide compressible, nous avons exploré la présence de phases 
non sphériques dans les couches profondes de la croûte interne, et nous avons 
trouvé la séquence sphères $\rightarrow$ cylindres $\rightarrow$ plaques
$\rightarrow$ tubes pour chacun des quatre modèles considérés : BSk24, SLy4, 
BSk22, et DD-ME$\delta$. Nous avons montré que le point de transition vers la 
matière homogène est très sensible au comportement de la tension de surface
pour des valeurs extrêmes de l'isospin, qui est controlé par le paramètre de 
surface isovecteur $p$.
%
Malheureusement, il est impossible d'accéder à la valeur de ce paramètre à 
partir des données empiriques actuelles de physique nucléaire 
qui sont limitées à $I \lesssim 0.3$. Nous avons vu que 
les résultats de la littérature sur la densité et la pression de transition
croûte-coeur pour la dynamique spinodale~\cite{Ducoin2011} sont globalement 
bien reproduites par notre calcul depuis la croûte avec $p \approx 3$. Nous 
avons également confirmé la corrélation entre la densité de transition $n_t$ et 
le paramètre empirique $L_{sym}$, déjà observée dans de précédents travaux.

Nous avons déterminé les équations d'équilibre caractérisant l'état fondamental 
de la matière dans le coeur, qui consiste en de la matière \textit{npe$\mu$} 
jusqu'à $n_B\approx 2n_{sat}$. La forte corrélation entre l'énergie de symétrie 
et la fraction de proton a été expliquée. 
Nous avons constaté que des muons apparaissent à approximativement 
$n_B = 0.12$ fm$^{-3}$ pour chaque modèle. Comme dans de précédents travaux 
\cite{Wiringa1988,Douchin2001}, nous avons extrapolé la matière 
\textit{npe$\mu$} à des densités plus élevées, puisque les interactions 
hyperon-hyperon et hyperon-nucléon restent actuellement très peu connues.

Enfin, nous avons souligné que l'utilisation d'une équation d'état unifiée 
est essentielle pour déterminer les observables de la croûte. En ce sens, nous 
avons proposé une métamodélisation de l'équation d'état pour une étoile à 
neutrons froide non accretante, où la croûte et le coeur sont traités de 
manière uniforme, c'est-à-dire avec les mêmes paramètres empiriques. 
En utilisant les paramètres empiriques de SLy4, nous avons montré que la 
différence relative avec l'équation d'état de Douchin et Haensel est inférieure 
à $10\%$ dans la croûte interne et à $1\%$ dans le coeur.

\section{Chapitre 2}

Dans ce chapitre, nous avons d'abord introduit les équations de base de
l'équilibre hydrostatique en relativité générale, pour les étoiles sphériques 
non rotatives. Nous avons résolu ces équations pour différentes équations
d'état populaires afin d'obtenir la relation entre la masse et le rayon de 
l'étoile. 
Dans le même esprit, nous avons calculé le moment d'inertie ainsi que la 
déformabilité due aux effets de marée, qui décrit la rigidité de l'étoile
soumise aux tensions causées par la force gravitationnelle exercée par une
étoile secondaire, dans l'approximation de la rotation lente, qui devrait être 
valable pour la plupart des pulsars. 
La détermination de la densité et de la pression au point de transition entre
la croûte et le coeur, discutée dans le premier chapitre, permet le calcul des
observables de la croûte. Nous avons ainsi calculé son épaisseur et 
la fraction du moment d'inertie résidant à l'intérieur. Nous
avons expliqué en détail le lien entre le moment d'inertie de la croûte et le 
phénomène de glitch de pulsar dans le consensus actuel, c'est-à-dire où un 
glitch est observé lors du transfert de moment angulaire depuis les neutrons 
superfluides, confinés dans la croûte interne, vers le reste de 
l'étoile~\cite{Anderson1975}.
%
L'un des principaux problèmes exposés dans la section~\ref{sec:tov} est la 
sensibilité des résultats à l'équation d'état, et plus particulièrement aux
paramètres empiriques du secteur isovecteur. Nous avons montré que toutes les 
équations d'état considérées ne réussissent pas à dépasser la contrainte de 
masse maximale, $M_{max} \gtrsim 
2M_\odot$ \cite{Demorest2010,Antoniadis2013,Cromartie2020}, en faveur d'une
équation d'état \textit{stiff}. 
De plus, nos résultats tendent à indiquer que le phénomène de glitch ne peut 
pas uniquement tirer son origine de la physique de la croûte compte tenu des 
estimations actuelles de l'effet 
d'entrainement~\cite{Andersson2012,Piekarewicz2014}.
Inversement, la récente contrainte sur le paramètre de déformabilité due aux
effets de marée $\Lambda_{1.4}$ déduit de GW170817~\cite{GW1} tend à favoriser 
les équations d'état \textit{soft}.

Nous avons rappelé le principe de l'inférence bayésienne avant d'effectuer la  
détermination bayésienne des paramètres de l'équation d'état en utilisant 
technique de métamodélisation~\cite{Margueron2018a}. Nous avons utilisé une 
distribution \textit{prior} uniforme pour les paramètres qui tient compte tenu 
des contraintes empiriques actuelles. 
%
Les résultats et les analyses présentés dans ce chapitre sont proches, mais ne 
sont pas identiques, à nos résultats 
publiés~\cite{Carreau2019cc,Carreau2019moi}. 
En effet, une analyse bayésienne dépend de façon cruciale à la fois du
\textit{prior} et de la modélisation de le fonction de vraisemblance. 
Dans ce travail de thèse, nous avons fait varier à la fois la distribution
\textit{prior} des paramètres de surface, en adoptant un protocole de fit 
différent de celui de~\cite{Carreau2019cc,Carreau2019moi} et en introduisant 
des termes de courbure qui ont été négligées dans l'étude précédente, et les 
filtres, en explorant les effet de l'inclusion ou non des prédictions de la 
théorie des perturbations chirales concernant la pression de la matière 
nucléaire symétrique et la matière pure de neutrons.
Les différents résultats sont compatibles à l'intérieur des barres d'erreur, 
mais différentes corrélations sont observées. Nous avons analysé ces
différences, ce qui nous a permis de mieux comprendre l'effet des différentes 
conditions.
%
Une analyse de sensitivité sur le point de transition croûte-coeur a révélé que 
les paramètres empiriques du secteur isovecteur sont les plus important pour 
la déterminer précisément la densité $n_t$ et pression $P_t$ de transition. 
Nous avons calculé la fonction de vraisemblance en tenant compte de 
contraintes sur les observables de physique nucléaire (contraintes de basse
densité), à savoir les masses expérimentales~\cite{Huang2017} 
et les calculs en théorie des perturbations chirales pour la matière nucléaire
symétrique et la matière pure de neutrons~\cite{Drischler2016}, et sur les 
observables des étoiles à neutrons ainsi que des exigences de base en physique
(contraintes de haute densité). 
La distribution \textit{posterior} des paramètres empiriques a été analysée. 
Nous avons observé que le filtre de basse densité est très efficace pour 
contraindre les paramètres dans le secteur isovecteur, en particulier si l'on 
impose la compatibilité avec les prédictions de la théorie des perturbations
chirales pour la pression de la matière nucléaire. 
Nous avons montré que seules les contraintes en physique nucléaire donnent lieu 
à des corrélations entre les paramètres empiriques. 
Dans le secteur isovecteur, la corrélation entre $E_{sym}$ et $L_{sym}$ est 
retrouvée, et nous avons constaté que le paramètre $L_{sym}$ est corrélé avec 
$K_{sym}$. Nous avons trouvé des nouvelles corrélations intéressantes lors de 
l'ajout du filtre sur la pression : $r(L_{sym},Q_{sym})=-0.55$ et 
$r(K_{sym},Q_{sym})=0.52$.

Enfin, nous avons fait des prédictions générales sur les propriétés statiques 
et les observables de la croûte des étoiles à neutrons, en utilisant la 
distribution \textit{posterior} des paramètres calculée dans la 
section~\ref{sec:bayes}. Dans l'ensemble, nous avons constaté que nos 
prédictions pour les observables de l'étoile sont en assez bon accord avec les
les contraintes sur l'équation d'état, le rayon et la déformabilité due aux
effets de marée, déduites de l'événement GW170817~\cite{De2018,GW1}. Nous avons 
obtenu $R_{1.4}=12.88_{-0.65}^{+0.53}$ km et 
$\Lambda_{1.4} = 634_{-190}^{+204}$ avec un intervalle de confiance de $90\%$. 
%
Nous avons montré que la densité et la pression de transition sont 
corrélées avec les paramètres empiriques du secteur isovecteur, notamment avec 
$Q_{sym}$. En considérant l'incertitude expérimentale et théorique 
sur ces paramètres, nous avons estimé la densité et pression de transition
respectivement à $n_t=0.068_{-0.021}^{+0.021}$ fm$^{-3}$ et
$P_t=0.263_{-0.149}^{+0.302}$ MeV/fm$^3$ à $1\sigma$. Contrairement à nos 
travaux publiés~\cite{Carreau2019cc}, nous n'observons pas une forte influence
des paramètres de surface sur les quantités au point de transition 
croûte-coeur. Cela peut s'expliquer par un contrôle plus important de l'énergie 
de surface dans cette thèse. 
%
Nous avons observé que plus la croûte est épaisse, plus la quantité de moment 
angulaire confinée dans le superfluide coexistant avec le réseau d'ions 
dans la croûte interne est importante. 
La grande incertitude sur la pression de transition se reflète sur le moment
d'inertie de la croûte. En effet, pour un étoile à neutrons de $1.4M_\odot$, 
nous avons obtainu $I_{crust,1.4}/I_{1.4}=2.89_{-1.68}^{+2.51} \%$ à $1\sigma$. 
La distribution de probabilité pour la masse et le rayon de Vela a été calculée 
pour deux estimations différentes de l'effet d'entraînement de la croûte. 
Ceci nous a permis de confirmer qu'une meilleure estimation quantitative de
l'effet d'entraînement de la croûte est le point clé pour associer le phénomène 
de glitch à la physique de la croûte, même si l'incertitude sur l'équation
d'état brouille les résultats. Nous nous sommes intéressés à la variation du 
moment d'inertie de la croûte avec la masse de l'étoile et avons conclu que le 
phénomène de glitch démontré par certains pulsars ne peut pas uniquement tirer 
son origine de la physique de la croûte dans le cas où l'on considère la plus 
grande estimation actuelle de l'effet d'entraînement~\cite{Delsate2016}.

\section{Chapitre 3}

Dans ce dernier chapitre, nous nous sommes intéressé à la modélisation de la 
croûte à température finie. 
L'approximation de plasma à un composant, dans laquelle la distribution 
attendue de noyaux est remplacé par un noyau unique déterminé à partir de la 
minimisation du potentiel thermodynamique à une condition thermodynamique 
donnée, a été considérée. 
Nous avons donné les expressions des corrections entrant en jeu dans l'énergie 
libre des ions à température finie, à savoir le terme de mouvement de 
translation du centre de masse dans le liquide, le terme de vibrations 
quantiques de point zéro dans la phase solide ainsi que la contribution de 
l'interaction coulombienne qui diffère selon la phase de matière. 
Nous avons discuté la définition de la transition de la phase liquide à la
phase solide, puis nous avons inclus la distribution des ions en considérant un 
plasma à plusieurs composants à l'équilibre. 
%
Dans ce cadre, des effets de mélange non linéaires apparaissent car la 
position du centre de masse de chaque ion n'est pas limitée au seul volume de 
la cellule de Wigner-Seitz associée mais peut explorer librement le 
volume total. 
Ceci est connu dans la littérature de la physique des plasmas sous le terme 
d'\guillemotleft entropie de mélange\guillemotright.
Les potentiels chimiques des neutrons et des protons dans le mélange se sont 
avérés être très proches de ceux dans le plasma à un composant, ce qui montre 
qu'une implémentation perturbative de l'équilibre statistique nucléaire est 
suffisante.
%
Nous avons vu qu'un terme de réarrangement entre dans l'expression du 
potentiel thermodynamique de part l'auto-cohérence induite par la partie 
coulombienne de l'énergie libre de l'ion. 
Nous avons proposé une approximation de ce terme pour éviter de 
résoudre un problème auto-cohérent complexe en imposant la coïncidence du noyau
le plus probable avec la solution obtenue dans la limite du plasma à un 
composant. 
Comme lors de notre étude à température nulle, nous avons choisi d'utiliser 
l'approche de la goutte liquide compressible pour modéliser l'énergie libre des
clusters dans le régime des neutrons libres. 
L'expansion de Sommerfeld, fiable à basse température, a été employée pour 
dériver l'expression de l'énergie libre de le matière nucléaire ainsi que 
du potentiel chimique des nucléons. 
En raison des faibles valeurs de la température de cristallisation prédites 
dans la croûte, nous avons négligé l'excitation des modes de surface.

En utilisant les masses expérimentales complétées par des tables de masses 
théoriques basée sur le modèle microscopique Hartree-Fock-Bogoliubov, nous 
avons obtenu des résultats pertinents pour la croûte externe des protoétoiles à
neutrons, c'est-à-dire à basse température et plus particulièrement au point de 
cristallisation.
%
Nous avons calculé la température de cristallisation du plasma à un composant 
dans la croûte externe. Nous avons constaté que le fait de négliger la 
contribution anharmonique dans l'énergie libre des ions dans la phase solide 
conduit à une sous-estimation de la la température de cristallisation, qui 
varie d'environ $5 \times 10^7$ K à très basse pression à $2.8 \times 10^9$ K 
au voisinage de la limite de stabilité neutron et présente un comportement 
discontinu en raison des effets de couches. 
Nous avons vérifié que la température de cristallisation n'est pas fortement 
dépendante du modèle de masse utilité dans les couches profondes de la croûte 
externe, tant que des tables de masses théoriques réalistes sont utilisées pour
compléter les données expérimentales.
%
Nous avons constaté que la composition de la croûte externe au point de 
cristallisation est très proche de celle calculée dans l'hypothèse de matière
froide catalysée, et que la valeur moyenne du nombre de masse $A$ et de charge 
$Z$ dans le plasma à plusieurs composants est très proche de la solution pour
le plasma à un composant dans la croûte externe. 
À mesure que la pression et la température augmentent, nous avons montré que la 
distribution de charges est plus étendue, indiquant alors que l'approximation 
du plasma à un composant devient moins fiable. 
%
Nous avons calculé le paramètre d'impureté $Q_{\text{imp}}$, qui 
représente la variance de la distribution de charges, à la température de 
cristallisation et nous avons retrouvé les résultats de~\cite{Fantina2020}.
D'importantes oscillations de $Q_{\text{imp}}$ ont été observées, ce qui 
suggère que la croûte externe pourrait consister en une alternance de 
couches hautement isolantes et de couches hautement conductrices et nous 
avons constaté que le facteur d'impureté pouvait atteindre des valeurs aussi 
élevées que $\approx 50$ lorsque la distribution de charges présente un 
caractère multimodal. 
%
Nous avons également évalué la fraction des noyaux de masse impaire et de 
charge impaire présents dans la croûte externe. Nous avons montré que ces 
quantités augmentent avec la température, et qu'au point de cristallisation, 
les noyaux impairs constituent environ $2\%$ des espèces de la croûte externe 
et contribuent à hauteur d'environ $2.4\%$ de la masse baryonique de la croûte 
externe. 

Enfin, en utilisant les dernières fonctionnels de BSk, nous avons obtenu des 
résultats pertinents pour la croûte interne des protoétoiles à neutrons, en 
particulier à la température de cristallisation.
%
Suite à notre travail à température nulle, nous avons ajouté des 
les corrections de couches à l'énergie libre des clusters. En se plaçant 
dans l'approximation de plasma à un composant, nous avons observé que la plus 
grande source de dépendance au modèle de la température de cristallisation et 
composition associée provient de la une partie lisse de la fonctionnelle 
nucléaire. 
En outre, nous avons vu que les effets de couches sont fortement atténués à la 
température de cristallisation dans la croûte interne, que nous avons estimé 
entre $\approx 3 \times 10^9$ K et $\approx 9 \times 10^9$ K. 
En ce qui concerne la composition à la cristallisation, nous avons observé, à 
faible densité, des écarts par rapport à la matière froide catalysée. 
%
Nous avons montré qu'à la 
température de cristallisation la composition de la croûte interne est 
dominée par les clusters avec un nombre de charge $Z\approx 40$, tandis que la 
distribution de charges s'étend d'environ $20$ proche de la limite de stabilité 
neutron, à environ $40$ près du point de transition vers la matière homogène. 
Cela se reflète dans le comportement du paramètre d'impureté qui de manière 
monotone augmente avec la densité baryonique moyenne jusqu'à environ $40$ dans 
les couches les plus profondes de la croûte interne. 
Nous avons également observé que l'inclusion du terme de réarrangement est 
nécessaire pour garantir la cohérence thermodynamique. 

\section{Conclusions générales et perspectives}

L'objectif principal de cette thèse a été de fournir des prédictions réalistes 
et d'étudier les sources d'incertitudes sur les observables des étoiles à 
neutrons froides isolées et des protoétoiles à neutrons chaudes, en utilisant 
une métamodélisation unifiée pour décrire l'énergétique de la matière d'étoile, 
qui permet de tenir compte des contraintes actuelles liées aux expériences 
nucléaires et des observations astrophysiques.
\\

Nous avons envisagé une approche de métamodélisation unifiée afin de 
calculer la composition et l'équation d'état de l'étoile à neutrons froide
isolée pour toute fonctionnelle de matière nucléaire en utilisant les 
paramètres empiriques associés comme seuls ingrédients. Nous avons obtenu une
erreur qui, dans le cas de l'équation d'état basée sur SLy4, est inférieure à 
$1\%$. 
Pour modéliser l'énergie des clusters dans la croûte interne, nous avons 
proposé une version du célèbre modèle de la goutte liquide compressible basée 
sur la technique de métamodélisation, avec un paramétrage de la tension de 
surface suggéré par des calculs Thomas-Fermi à des ratios d'isospins extrêmes. 
%
La composition obtenue dans l'état fondamental suit de près les résultats de 
calculs Thomas-Fermi étendu rapportés dans la littérature. Le principal 
inconvénient de l'approche de la goutte liquide compressible est que les effets 
quantiques sont perdus. Cependant, nous avons montré que les nombres magiques 
dans la croûte interne peuvent être retrouvés en ajoutant les corrections de 
couches calculées par la méthode de Strutinsky, conduisant alors à un très bon 
accord avec les résultats de calculs Thomas-Fermi étendu avec corrections de
couches Strutinsky. 
%
La même séquence de phases non sphériques dans les couches les plus profondes 
de la croûte interne a été observée pour tous les modèles considérés, mais nous 
avons souligné qu'elle est sensible au comportement de la tension de surface à 
grand isospin, qui reste peu contrainte à ce jour. 
%
L'incertitude sur l'équation d'état induite par le traitement de l'énergie de 
surface peut être estimé à $10\%$, ce qui correspond à la différence entre 
nos résultats avec la fonctionnelle SLy4 et l'équation d'état de Douchin et
Haensel qui utilise la même fonctionnelle mais un traitement différent de 
l'énergie de surface. 
Il en va de même pour l'emplacement du point de transition vers 
la matière homogène $npe$, qui a été calculée pour plusieurs modèles nucléaires
depuis la croûte. 
%
Nous avons montré que l'anticorrélation entre la pente de l'énergie de symétrie
à la densité de saturation $L_{sym}$ et la densité de transition croûte-coeur, 
signalée dans de nombreux travaux antérieurs, est obtenue si l'énergie de 
surface est seulement optimisée sur un ensemble limité de noyaux magiques et 
semi-magiques sphériques. Inversement, si la tension de surface est optimisée 
sur un grand nombre de données et si des termes de courbure sont ajoutés,
cette corrélation disparaît et la sensibilité à l'équation d'état est limitée
aux paramètres empiriques d'ordres supérieurs dans le secteur isovecteur
($K_{sym}$, $Q_{sym}$), qui restent peu contraints à ce jour.

Nous avons exploité le principal atout de la technique de métamodélisation, à 
savoir le fait qu'aucune corrélation artificielle n'est introduite a priori 
entre les paramètres empiriques, pour effectuer la détermination bayésienne ces
derniers, conduisant à des prédictions réalistes pour les observables des 
étoile à neutrons. Nous avons considéré a priori une distribution uniforme pour 
les paramètres empiriques dont les limites sont compatible avec les 
contraintes expérimentales. 
%
La fonction de vraisemblance que nous avons construite tient compte de 
récents résultats de la théorie des perturbations chirales pour la matière 
nucléaire symétrique et la matière pure de neutrons jusqu'à 0.20 fm$^{-3}$,
ainsi que de la contrainte de masse maximale de l'étoile et des principes
physiques de base. 
Elle comprend également une probabilité dans laquelle est encodée la capacité 
du modèle de goutte liquide compressible à reproduire les masses de l'AME2016. 
Nous avons montré que le fait d'imposer les contraintes 
associées aux calculs ab initio est très efficace pour contraindre les 
paramètres empiriques dans le secteur isovecteur, et que cela donne lieu à des 
corrélations entre les dérivés de l'énergie de symétrie à la densité de 
saturation.
Nous avons fait des prédictions générales pour les propriétés statiques en 
utilisant la distribution \textit{posterior} des paramètres empiriques et avons 
constaté que nos résultats sont compatibles avec les contraintes des 
collaborations LIGO et Virgo déduites de l'événement GW170817.
%
La seule hypothèse de la technique de métamodélisation étant la 
possibilité d'étendre l'équation d'état en une série de Taylor, ce résultat 
implique que nous n'avons pas de preuve irréfutable d'une transition de la 
phase de premier ordre dans le coeur des étoiles à neutrons, bien que cela
ne puisse évidemment pas être exclu.
%
Le moment d'inertie de la croûte, qui est fortement corrélé avec la 
localisation du point de transition croûte-coeur, a été calculé, et les
résultats tendent à indiquer que le phénomène de glitch ne peut pas uniquement
tirer son origine de la physique de la croûte. 
Cela ouvre des possibilités intéressantes concernant la pertinence des 
composantes superfluides dans le coeur, en particulier dans le canal $^3P_2$ 
pour les paires $nn$ et $pp$.

Dans la continuité d'un travail récent sur la croûte externe, nous avons 
envisagé un équilibre statistique complet des ions dans la croûte à température 
finie permettant la présence d'un gaz de neutrons. Nous avons évalué 
l'abondance des noyaux de masse impaire et des noyaux de charge impaire 
présents dans la croûte externe à la température de cristallisation. 
Leur présence est intéressante, car elle pourrait causer des transitions de
phase ferromagnétiques.
%
Nous avons considéré une dépendance en température des corrections de couches 
des protons dans le régime de la croûte interne. La température de 
cristallisation et la composition associée ont été calculées dans 
l'approximation de plasma à un composant, et nos résultats suggèrent que la 
source la plus importante de dépendance au modèle provient de la partie lisse 
de la fonctionnelle, l'ingrédient le plus important à fixer pour une la 
prédiction quantitative des propriétés de la croûte interne étant la tension de 
superface à des ratios d'isospins extrêmes. 
%
Des écarts par rapport à la matière froide catalysée ont été observés à faible 
densité et température de cristallisation, ce qui pourrait avoir un impact sur 
les simulations du processus $r$.
Enfin, nous avons calculé le paramètre d'impureté de manière systématique dans
la croûte interne au point de cristallisation pour quatre fonctionnelles BSk
récentes qui tiennent compte des incertitudes actuelles sur l'énergie de 
symétrie. 
Il s'agit à ce jour du premier calcul du facteur d'impureté dans la croûte
interne basé sur des fonctionnelles nucléaires réalistes, et il a montré 
que la contribution des impuretés est non négligeable, ce qui pourrait modifier 
les propriétés de transport dans la croûte.
\\

Il est clair que la détermination expérimentale des paramètres empiriques
d'ordres supérieures du secteur isovecteur dans les prévisions théoriques de 
la théorie des perturbations chirales à basse densité et dans la modélisation
microscopique de l'énergie de surface à des ratios d'isospins extrêmes 
sont nécessaires pour réduire les incertitudes des observables de la croûte des
étoiles à neutrons. 
%
De plus, nous avons montré que les contraintes sur les propriétés de la matière 
dense peuvent être déduites des observations astrophysiques dans un cadre 
bayésien. 
De nombreuses nouvelles mesures sont attendues dans un futur proche de la part
de NICER et de LIGO/Virgo. 
Nous pouvons donc espérer réduire les incertitudes sur les dérivés de l'énergie 
de symétrie d'ordres élevés et sur les observables de l'étoile en utilisant une 
méthode bayésienne ou l'apprentissage machine.

Diverses applications du formalisme introduit pour la description d'un plasma à
plusieurs composants dans un équilibre statistique complet peut être envisagé 
à l'avenir. 
Par exemple, il est envisageable d'étudier la présence d'hyperons dans la 
croûte au point de cristallisation. 
Les géométries non sphériques peuvent également être prises en compte dans ce
traitement, ce qui permettrait l'évaluation du paramètre d'impureté dans les 
couches profondes de la croûte, pouvant ainsi vérifier l'hypothèse 
de~\cite{Pons2013} que la présence d'une couche hautement résistive dans la 
croûte interne pourrait conduire à une limite plus élevée de la période de 
rotation des pulsars X.

Durant cette thèse, j'ai écrit une bibliothèque open source en langage C,
NSEoS, dans le but de fournir des outils utiles liés à la physique des étoiles 
à neutrons~\cite{NSEoS}. La bibliothèque a été utilisée pour produire tous les 
résultats présentés dans cette thèse.
De plus, elle a déjà été utilisée par différents étudiants pendant leur stage 
sous la supervision conjointe de ma directrice de thèse et de moi-même, et 
nous pensons qu'elle peut servir de base à de futures études.
