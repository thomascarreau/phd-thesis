\appendix
\chapter{Energy density of a relativistic electron gas}\label{appendix:epse}

We give here the derivation of the energy density of a relativistic electron
gas of density $n_e$ at zero temperature.

From $\approx 10^{14}$ g/cm$^3$, electrons are essentially free. In this
case, the energy density is given by
%
\begin{equation}
  \varepsilon_e = \frac{c}{\pi^2}\int_{k=0}^{k_{e}} dk k^2 
  \sqrt{\hbar^2 k^2 + m_e^2c^2},
\end{equation}
%
$c$ being the speed of light, $\hbar$ the reduced Planck constant, and $m_e$
the electron mass. The electron Fermi wave number $k_e$ is related to the
electron density via $k_e = (3\pi^2n_e)^{1/3}$. Making a simple variable, 
$x = \hbar k/(m_ec)$, we can rewrite the energy density as
%
\begin{equation}
  \varepsilon_e = \frac{m_e^4c^5}{\pi^2\hbar^3}\xi(x_r)
\end{equation}
%
with $x_r = \hbar k_e/(m_ec)$ and
%
\begin{equation}
  \xi = \int_{x=0}^{x_r} dx x^2\sqrt{x^2+1}.
\end{equation}
%
Integrating by parts, it follows
%
\begin{eqnarray}
  \xi &=& \left[\frac{x^3}{3}\sqrt{x^2 + 1}\right]_0^{x_r} -
  \int_0^{x_r}dx\frac{x^4}{3\sqrt{x^2+1}}\notag\\
  3\xi &=& x_r^3\sqrt{x_r^2+1} 
  - \int_0^{x_r}dx x^2 \left(\frac{x^2+1}{\sqrt{x^2
  + 1}} - \frac{1}{\sqrt{x^2+1}}\right)\notag\\
        4\xi &=& x_r^3\sqrt{x_r^2+1} + \int_0^{x_r}dx \frac{x^2}{\sqrt{x^2+1}}.
\end{eqnarray}
%
We have
%
\begin{equation}
  \frac{d}{dx}\left(x\sqrt{x^2 + 1}\right) = \sqrt{x^2+1} 
  + \frac{x^2}{\sqrt{x^2+1}},
\end{equation}
%
and
%
\begin{equation}
  \frac{d}{dx}\left[\ln(x + \sqrt{x^2+1})\right] = \sqrt{x^2+1} -
  \frac{x^2}{\sqrt{x^2+1}}.
\end{equation}
%
thus,
%
\begin{equation}
  \frac{1}{2}\frac{d}{dx}\left[x\sqrt{x^2+1} -
  \ln\left(x+\sqrt{x^2+1}\right)\right] = \frac{x^2}{\sqrt{x^2+1}}.
\end{equation}
%
Hence,
%
\begin{equation}
  4\xi = x_r^3\sqrt{x_r^2+1} + \frac{1}{2}\left[x\sqrt{x^2+1} -
  \ln\left(x+\sqrt{x^2+1}\right)\right]_0^{x_r},
\end{equation}
%
which finally gives
%
\begin{equation}
  \varepsilon_e(n_e) = \frac{m_e^4c^5}{8\pi^2\hbar^3}\left[x_r(2x_r^2 +
  1)\gamma_r - \ln(x_r + \gamma_r)\right], 
\end{equation}
%
where we have introduced the parameter parameter $\gamma_r = \sqrt{x_r^2+1}$.

\chapter{Neutron and proton chemical potentials in the 
metamodel}\label{appendix:mutau}

From the metamodeling of the nuclear matter energy, Eq.~\ref{eq:meta}, we give 
here the complete expressions of the neutron and proton chemical potentials
for a homogeneous system characterized by a neutron density $n_n$ and proton
density $n_p$ at zero temperature. Let us introduce the variables 
$x=(n-n_{sat})/(3n_{sat})$ and $\delta = (n_n - n_p)/n$, $n=n_n+n_p$ being the
total baryon density.

The neutron chemical potential, with the rest mass energy $m_nc^2$, is given by
%
\begin{eqnarray}
  \mu_{HM,n} &=& e_{HM}(n_n,n_p) + n\left(\frac{\partial
  e_{HM}}{\partial n_n}\right)_{n_p} + m_nc^2\\
  \mu_{HM,n} &=& e_{HM}(n,\delta) + \frac{1+3x}{3}\left(\frac{\partial
  e_{HM}}{\partial x}\right)_\delta + (1 - \delta)\left(\frac{\partial
e_{HM}}{\partial \delta}\right)_x + m_n c^2.
\end{eqnarray}
%
Equivalently, the proton chemical potential is given by
%
\begin{equation}
  \mu_{HM,p} = e_{HM}(n,\delta) + \frac{1+3x}{3}\left(\frac{\partial
  e_{HM}}{\partial x}\right)_\delta - (1 + \delta)\left(\frac{\partial
e_{HM}}{\partial \delta}\right)_x + m_p c^2.
\end{equation}
%
We first go through the derivative with respect to $x$, yielding
%
\begin{eqnarray}
  \left(\frac{\partial e_{HM}}{\partial x}\right)_\delta &=&  
  \frac{5}{1+3x}t_{HM}^{FG}(n,\delta) \notag\\
                                                         &&-
  \frac{3}{1+3x}\frac{1}{2}t_{sat}^{FG}(1+3x)^{2/3}
  \left[(1+\delta)^{5/3}+(1-\delta)^{5/3}\right]\notag\\
                                                         &&+
  \sum_{\alpha \geq 0}
  (v_\alpha^{is} + v_\alpha^{iv}\delta^2)\frac{1}{\alpha!}\left(\alpha
  x^{\alpha-1}u_\alpha(x) + x^\alpha \frac{du_\alpha}{dx}\right).
\end{eqnarray}
%
Let us notice that is gives the expression for the total pressure,
%
\begin{eqnarray}
  P_{HM}(n,\delta) &=& n^2\left(\frac{\partial e_{HM}}{\partial
  n}\right)_\delta\\
  &=& \frac{1}{3}n_{sat}(1+3x)^2\left(\frac{\partial e_{HM}}{\partial
  x}\right)_\delta.
\end{eqnarray}
%
We now turn to the derivative with respect to the asymmetry $\delta$,
%
\begin{eqnarray}
  \left(\frac{\partial e_{HM}}{\partial \delta}\right)_x &=&
  \frac{1}{2}t_{sat}^{FG}(1+3x)^{5/3}\left[(1+\delta)^{5/3} 
  - (1-\delta)^{5/3}\right]\notag\\
                                                         &&+ 
  \frac{5}{6}t_{sat}^{FG}(1+3x)^{2/3}\left[(1+\delta)^{2/3}\frac{m}{m_n^*} -
  (1-\delta)^{2/3}\frac{m}{m_p^*}\right]\notag\\
                                                         &&+
  2\delta \sum_{\alpha \geq 0}v_\alpha^{iv}\frac{x^\alpha}{\alpha!}u_\alpha(x).
\end{eqnarray}
%

\chapter{Partie en français}\label{appendix:fr_part}

Dans cette annexe sont données les traductions en français de l'introduction 
générale, de la conclusion générale et des conclusions partielles.

\section{Introduction générale}

L'existence des étoiles à neutrons a été proposée dès 1931 par
Landau~\cite{Landau1932}, un an avant la découverte du neutron par
Chadwick~\cite{Chadwick1932}. Landau avait anticipé que la matière stellaire
pouvait devenir \guillemotleft~très dense, au point que les noyaux des atomes 
soient presque en contact, formant ainsi un gigantesque noyau~\guillemotright.
En 1934, Baade et Zwicky introduisirent le terme \textit{supernova} pour
désigner un \guillemotleft~type de remarquable de nova géante~\guillemotright,
un phénomène rare et très énergétique caractérisé par un éclat soudain 
et éphémère de la luminosité suivi par un lent déclin, et prédirent que
\guillemotleft~les supernovas représentent la transition des étoiles ordinaires 
en étoiles à \textit{neutrons}~\guillemotright~\cite{Baade1934}.
La présence des étoiles à neutrons dans l'Univers est restée purement théorique
jusqu'à 1968, lorsqu'une source rapidement pulsante, un \textit{pulsar}, fut
observée pour la première fois par Jocelyn Bell, une étudiante en thèse 
supervisée par Antony Hewish~\cite{Hewish1968}. Plusieurs semaines après cette
observation et suite à la découverte du pulsar du Crabe en 1968 (qui ne 
pouvait pas correspondre à une naine blanche étant donné sa période de
pulsation trop courte~\cite{Comella1969}), les pulsars furent identifiés comme
des \guillemotleft~des étoiles à neutrons rotatives~\guillemotright par
Gold~\cite{Gold1968}, ouvrant alors la voie à des développements théoriques 
majeurs et à de nombreuses observations dans les décennies suivantes.
\\

Au cours de leur vie, les étoiles de masse supérieure à 8$M_\odot$ ($M_\odot$ 
étant la masse du Soleil) peuvent fusionner les éléments du c\oe ur jusqu'au 
silicium brûlant en fer. La fusion des éléments n'est alors plus possible 
dans la mesure où le fer est le nucléide le plus stable présent dans la nature. 
Dès lors, la chaîne de réactions dans le c\oe ur se termine, et dans leur stade 
final les étoiles présentent une structure en forme d'oignon : leur c\oe ur est 
composé de fer et de noyaux riches en neutrons du groupe du 
fer~\cite{Bethe1979}, entouré de couches d'éléments de moins en moins massifs 
jusqu'à l'hydrogène inerte à des températures et des densités plus 
basses~\cite{Woosley2002}. 
À ce stade, le c\oe ur stratifié est essentiellement soutenu par la 
pression de dégénérescence des électrons et sa masse ne cesse d'augmenter par 
accrétion, en brûlant les couches de silicium, jusqu'à atteindre la masse de 
Chandrashekar, $M_{Ch} \sim 1.44M_\odot$, à partir de laquelle la force 
gravitationnelle dépasse la pression de dégénérescence des 
électrons~\cite{Chandrasekhar1931}, déclanchant alors une explosion de 
supernova à effondrement de c\oe ur~\cite{Janka2007}. 
Le résidu de l'explosion est une protoétoile à neutrons chaude 
($\sim 10^{10}$~K) et deux finalités sont envisageables : soit la protoétoile à 
neutrons finira par évoluer en étoile à neutrons, soit en trou noir si sa 
masse est plus importante que la masse maximale des étoiles à neutrons, qui est 
encore incertaine à ce jour. 
L'évolution vers une étoile à neutrons froide prend environ cent ans, en 
passant par différentes étapes plus ou moins longues~\cite{Prakash1997}. 

L'étoile se refroidit en émettant des photons et des neutrinos. À environ 
$10^8$~K, la matière est catalysée, c'est-à-dire dans son état fondamental. 
Une représentation schématique de la structure interne d'une étoile à neutrons 
froide est donnée en fig.~\ref{fig:NStarInt}. 
% atmosphère
La surface externe d'une étoile à neutrons se compose d'une très fine 
atmosphère de seulement quelques centimètres et d'une enveloppe de quelques 
mètres, où le spectre du rayonnement électromagnétique thermique est formé. Ce 
rayonnement fournit des informations utiles sur les paramètres relatifs à la 
surface de l'étoile ainsi que sur les masses et rayons des étoiles à neutrons. 
% de la croûte
La croûte inhomogène est située sous la surface extérieure et fait 
environ $1$~km d'épaisseur. 
La croûte est généralement subdivisée en deux régions : la croûte externe et la 
croûte interne. La frontière entre ces deux régions se situe au niveau de
la surface où la limite de stabilité neutron est franchie, à quelques centaines 
de mètres sous l'atmosphère. 
Au sein de la croûte, les atomes sont entièrement ionisés et forment un réseau
cristallin immergé dans un gaz d'électrons relativistes ainsi que dans un gaz 
dans de neutrons si le potentiel chimique des neutrons est supérieur à la masse 
au repos des neutrons. 
De part la capture électronique, la matière s'enrichit en neutrons avec 
l'augmentation de la densité, c'est-à-dire lorsque l'on se rapproche du centre 
de l'étoile. 
Dans les couches profondes de la croûte interne, il est supposé que les noyaux 
présentent des formes non sphériques. 
%
À environ la moitié de la densité de saturation $n_{sat}$, correspondant à la 
la densité d'équilibre de la matière nucléaire homogène et symétrique, 
l'interface croûte-c\oe ur est atteinte et les noyaux disparaissent. 
Par analogie avec la croûte, nous pouvons distinguer le c\oe ur externe, 
correspondant aux densités baryoniques $0.5n_{sat} \lesssim n_B 
\lesssim 2n_{sat}$, et le c\oe ur interne, où $n_B \gtrsim 2n_{sat}$ ($n_B$
étant la densité de baryon). 
Dans le c\oe ur externe, la matière est constituée d'un mélange de neutrons, 
protons, électrons et éventuellement muons. La composition du c\oe ur interne 
est cependant incertaine et plusieurs hypothèses ont été avancées dans la
littérature, notamment l'apparence des hypérons, de condensats de bosons, et/ou 
une transition de phase vers les quarks. 
\\

Les pulsars ont été identifiés comme des étoiles à neutrons en rotation 
produisant des émissions pulsées, peu de temps après leur découverte fortuite 
en 1967 par Jocelyn Bell~\cite{Hewish1968}. Cinq décennies plus tard, nous en 
avons observé environ 2000 et de nombreuses techniques ont été développées pour 
mesurer leurs observables caractéristiques.

Il est plus facile pour les astronomes de mesurer la masse d'une étoile à 
neutrons appartenant à un système binaire. Il existe plusieurs types de 
binaires et, selon le type, différentes techniques sont utilisées 
pour déduire la masse de l'étoile~\cite{Haensel2007}.
% Effet de retard de Shapiro - MSP
Par exemple, dans un pulsar binaire, divers phénomènes tels que
l'effet Shapiro~\cite{Shapiro1964} peuvent être exploités dans le but de 
mesurer la masse en surveillant régulièrement la rotation d'un pulsar sur de 
longues périodes (années à décennies). L'effet Shapiro est un phénomène à
partir duquel des masses précises, tant pour un pulsar milliseconde que pour 
son compagnon, peuvent être déduites~\cite{Demorest2010,Cromartie2020}. 
Les masses mesurées des étoiles à neutrons par le \textit{timing} de pulsar 
sont affichées dans la fig.~\ref{fig:masses_lattimer}. 

Mesurer le rayon des étoiles à neutrons avec une grande précision est une tâche 
plus difficile. 
Le principe de base de l'extraction du rayon repose sur la mesure du 
spectre des rayons X (flux et fréquence), à partir duquel la température à la 
surface et le rayon de l'étoile peuvent être extraits à l'aide de l'équation 
du rayonnement du corps noir. 
En particulier, des estimations précises du rayon de l'étoile pourraient être 
fournies par des ajustements du spectre dans les binaires X de faible masse 
pendant les phases avec peu ou sans accrétion~\cite{Brown1998}.
%
Plusieurs binaires X de faible masse au repos ont été étudiés avec
l'observatoire de rayons X Chandra et/ou 
XMM-Newton \cite{Heinke2014,Servillat2012,Guillot2014,Guillot2013}.
%
Cependant, les résultats dépendent fortement des hypothèses faites sur la 
composition de l'atmosphère des étoiles à neutrons qui est peu 
connue à ce jour~\cite{Steiner2018}. 
Une amélioration importante est attendue avec l'analyse des résultats de la 
mission NICER, dont les premiers résultats commencent à être disponibles 
\cite{Bogdanov2019a,Bogdanov2019b,Miller2019,Raaijmakers2019,Riley2019}, 
même si des complications dans l'interprétation des données interviennent à
cause du manque d'uniformité de la température à la surface 
\cite{Bogdanov2019a,Bogdanov2019b,Miller2019,Raaijmakers2019,Riley2019}. 
Les valeurs canoniques typiques pour les masses et rayons des étoiles à 
neutrons sont $M = 1.4M_\odot$ et $R = 10-14$~km. 

Très récemment, la première détection d'ondes gravitationnelles à partir de la 
coalescence de deux étoiles à neutrons, l'événement GW170817, a donné des 
contraintes importantes pour la déformabilité due aux effets de 
marée~\cite{GW1}. 
La déformabilité décrit l'ampleur de la déformation d'un corps par les forces 
de marée qui surviennent lorsque deux corps massifs sont en orbite l'un autour 
de l'autre. L'exemple le plus simple et le plus connu correspond à la Lune qui 
provoque les marées observées dans les océans sur Terre. 
La détection du rayonnement gravitationnel émis par le binaire d'étoiles à
neutrons a été rendu possible par les détecteurs d'ondes gravitationnelles 
terrestres LIGO et Virgo. 
Peu avant la fusion, lorsque la distance relative entre les étoiles est
suffisamment faible, la distortion due aux effets de marée devient si 
importante que, dans certains cas (pour les signaux les plus forts 
correspondant aux événements les plus proches), il est possible de 
déduire la déformabilité à partir du signal. 

Le phénomène de \textit{glitch} observé chez certains pulsars correspond à un 
sursaut soudain dans la fréquence de rotation de l'étoile compacte. Ce sursaut 
pourrait s'expliquer par un transfert abrupt de moment angulaire 
depuis les composants superfluides de l'étoile vers la croûte solide de 
l'étoile et tous les composants normaux des fluides qui sont fortement couplés 
à la croûte par dissipation mutuelle. 
Il a été suggéré que ce transfert soudain soit dû au détachement des vortex 
superfluides confinés dans le réseau cristallin~\cite{Anderson1975}. 
%
En effet, un superfluide en rotation, tel que les neutrons superfluides à 
l'intérieur de la croûte, produit des vortex individuels quantifiés, avec une 
densité proportionnelle au taux de rotation. 
En raison de la force centrifuge associée à la rotation de l'étoile, ces vortex 
migrent vers la surface et se fixent aux ions du réseau constituant la croûte 
solide. Le ralentissement de la rotation de l'étoile causé par 
l'émission de rayonnement électromagnétique induit un décalage différentiel 
entre les vortex superfluides et la croûte, plus lente, entraînant
ainsi une tension. 
%
Lorsque le décalage entre la croûte solide et les vortex superfluides dépasse
un certain seuil et ne peut plus être soutenu, les vortex 
se détachent soudainement des sites du réseau, cédant alors du moment 
angulaire à la croûte et au reste de l'étoile qui est y mêlé par 
friction mutuelle, rétablissant des conditions proches de l'équilibre entre les 
composants normales et superfluides. 
Étant donné que le ralentissement dû à l'émission de rayonnement 
électromagnétique est un processus continu, il ne s'agit pas d'une situation 
d'équilibre finale. La tension finira ainsi éventuellement par se reconstituer, 
provoquant alors un autre \textit{glitch}. 
%
À ce jour, 555 \textit{glitches} ont été observés dans 190 pulsars via 
le \textit{timing} de pulsar de haute précision~\cite{Espinoza2011,Glitches}. 
Le pulsar de Vela (PSR B0833-45) est connu pour être le siège d'un grand nombre 
de \textit{glitches}, avec des évènements se produisant quatre fois par 
décennie en moyenne. 
\\

Une description théorique des différents phénomènes mentionnés ci-dessus 
nécessite la modélisation de la matière baryonique dense, en particulier de 
l'équation d'état nucléaire. 
%
L'équation d'état relie, dans des conditions connues de température et 
de densité, les quantités macroscopiques de l'étoile telles que la densité
massique et la pression. Elle permet notamment de déterminer la relation 
masse-rayon des étoiles à neutrons, obtenue en résolvant l'équation 
d'équilibre hydrostatique en relativité 
générale~\cite{Tolman1939,Oppenheimer1939}.
%
Étant donné que la chromodynamique quantique ne peut pas être exactement 
résolue dans le régime non perturbatif, l'équation d'état est fortement 
dépendante du modèle, ce qui induit des incertitudes dans la prévision des 
observables astrophysiques. 
Dans cette thèse, nous nous intéressons particulièrement à la modélisation de 
la croûte des (proto)étoiles à neutrons, où la matière est inhomogène. Du point 
de vue de la modélisation, ce régime de matière sous-saturée constitue la 
partie la plus délicate de l'équation d'état nucléaire. 
En effet, les incertitudes ne concernent pas 
seulement la fonctionnelle d'énergie nucléaire mais aussi la méthode
\textit{many-body} utilisée pour modéliser la matière inhomogène. 
%
En effet, l'évaluation de l'équation d'état implique de connaître la 
composition microscopique à chaque point de l'étoile. 
À température finie, une complication supplémentaire se présente de part le 
traitement statistique du problème. 
Historiquement, l'équation d'état de la matière sous-saturée a d'abord été 
calculée dans le cadre de l'approximation de noyau 
unique~\cite{BBP,Negele1973}, basée sur l'hypothèse que la matière 
peut être représentée par le noyau le plus probable, déterminé par la 
minimisation de la densité d'énergie libre de la matière. Bien que cette 
approximation soit exacte à température nulle, une distribution complète des 
\textit{clusters} doit être envisagée à température finie, comme c'est le cas 
dans les modèles d'équilibre statistique nucléaire. 
Une fois encore, la solution exacte du problème \textit{many-body} à 
température finie étant hors de portée, la modélisation ne peut pas être 
évitée, ce qui implique une dépendance au modèle lors du calcul des 
observables. 

Nous pourrions naïvement considérer qu'un modèle nucléaire optimal peut être 
extrait en confrontant les prédictions théoriques aux données 
observationnelles. Or, un problème majeur avec les équations d'état 
caractéristiques est que certaines observables sont plus fidèlement reproduites 
par un modèle particulier (ou une classe de modèles), tandis que ce dernier ne
parvient pas forcément à reproduire d'autres observables. 
En outre, chaque observable est associée à des incertitudes et la capacité de 
reproduire ou non une mesure déterminée n'a pas le même impact sur la 
fiabilité du modèle en fonction de l'observable considéré. 
En plus des contraintes astrophysiques, nous devons aussi tenir compte des 
contraintes fournies par les expériences de physique nucléaire et des 
développements récents dans les calculs \textit{ab initio} basés sur la théorie 
des perturbations chirales~\cite{Drischler2016}, consistant en la détermination 
de la fonctionnelle d'énergie nucléaire à partir d'une série entière respectant 
les symétries fondamentales de chromodynamique quantique~\cite{Machleidt2011} 
de basse énergie, c'est-à-dire la théorie des interactions fortes. 
Les différents modèles pouvant être envisagés n'ont pas été systématiquement 
confrontés à toutes ces contraintes. Dans ces conditions, il est très 
difficile de valider (ou d'invalider) un modèle. Une solution à cette impasse 
est assurée par l'utilisation du principe d'inférence bayésienne qui permet de 
mettre à jour nos croyances antérieures sur l'équation d'état en utilisant
les contraintes découlant des multiples sources mentionnées précédemment.
%
Il a été démontré que les observables des (proto)étoiles à neutrons sont 
sensibles à la microphysique entrant en jeu dans l'équation d'état, notamment 
aux dérivés d'ordres élevés de l'énergie de symétrie nucléaire et aux 
propriétés de surface des noyaux finis. 
Il est donc essentiel de contraindre ces quantités afin de contrôler les 
incertitudes sur les observables. 
Dans cette thèse, nous proposons de faire des prédictions réalistes et 
d'étudier les sources d'incertitudes associées aux observables des étoiles à
neutrons froides non accrétantes et des protoétoiles à neutrons chaudes, en 
utilisant les contraintes actuelles fournies par les expériences de physique 
nucléaire, les développements en théorie des perturbations chirales et les 
observations astrophysiques. 
Cet argument général s'applique à la modélisation totale de l'étoile à
neutrons mais aussi à la croûte de l'étoile, qui constitue l'objet principal de 
ce travail de thèse.

Alors que la croûte ne représente qu'un très faible pourcentage de la masse 
totale d'une étoile à neutrons, il est important de la modéliser correctement
s'il on veut comprendre la dynamique de ces astres compacts, notamment le
phénomène de \textit{glitch} et les processus de refroidissement. En plus de 
l'équation d'état nucléaire, la détermination des observables de la croûte 
nécessite la connaissance de la densité et de la pression au point de 
transition entre la croûte solide et le c\oe ur liquide~\cite{Piekarewicz2014}. 
Afin de valider l'hypothèse que le phénomène de \textit{glitch} tire son 
origine de la physique de la croûte, cette dernière doit être suffisamment 
épaisse pour contenir sufisamment de moment angulaire. Pour le pulsar de Vela, 
la fraction nécessaire de moment d'inertie contenu dans la croûte est 
actuellement estimée entre $1.6\%$ et 
$15\%$ \cite{Link1999,Andersson2012,Delsate2016}, selon 
l'importance de l'effet d'entraînement de la croûte, encore
activement débatu~\cite{Martin2016,Watanabe2017}. 
Une estimation fiable de l'épaisseur de la croûte et du moment d'inertie 
associé est donc indispensable. 
%
Pour toutes ces applications, il est essentiel de disposer de critères 
objectifs permettant de valider ou d'invalider les différents modèles, et 
éventuellement de corréler l'incertitude résiduelle des prédictions du modèle 
à des paramètres bien définis qui peuvent être contraints à l'avenir par des 
expériences plus précises ou des calculs \textit{ab initio}. 
La première partie de cette thèse visera a fournir un effort dans cette 
direction. Cela sera rendu possible par l'introduction d'une procédure de 
métamodélisation flexible qui nous permettra de confronter un très large 
ensemble de modèles de matière nucléaire catalysée aux différentes contraintes 
provenant de la physique nucléaire de basse énergie et des observations 
astrophysiques d'étoiles à neutrons matures. 

La deuxième partie de cette thèse portera sur la modélisation de la croûte à 
température finie. Une fois de plus, l'accent sera mis sur la détermination 
de barres d'erreur fiables sur les observables astrophysiques en raison des 
incertitudes inhérentes à la modélisation. 
Cette modélisation à température finie n'est pas seulement essentielle pour 
décrire l'évolution des supernovas et la dynamique des protoétoiles à neutrons 
mais pourrait également être pertinente pour la prédiction des observables 
de la croûte des étoiles à neutrons matures. 
En effet, il est peu probable que la croûte d'une étoile à neutrons soit à 
l'équilibre thermodynamique complet à température nulle : les étoiles à
neutrons naissent chaudes et si leur c\oe ur se refroidit suffisamment 
rapidement, la composition pourrait être gelée à une température 
finie~\cite{Goriely2011}. 
Des écarts par rapport à la composition de la croûte refroidie dans l'état 
fondamental autour de la limite de stablité neutron ont déjà été considérées
dans~\cite{Bisnovaty1979}. Toutefois, de simples extrapolations de formules de
masse semi-empiriques avaient été utilisées à cette époque. 
La valeur de la température à laquelle la composition se fige est difficile 
à évaluer. Cependant, une limite inférieure est donnée par la température de 
cristallisation car nous pouvons nous attendre à ce que les réactions 
nucléaires soient entièrement inhibées dans un cristal de Coulomb. 
Pour ces raisons, la dernière partie de cette thèse sera consacrée à l'étude 
de la structure de la croûte à la température de cristallisation. 
La présence éventuelle de phases amorphes et hétérogènes dans la croûte 
interne d'une étoile à neutrons devrait réduire la conductivité électrique de 
la croûte, avec des conséquences potentiellement importantes sur l'évolution 
magnéto-thermique de l'étoile. Dans les simulations de refroidissement, le 
désordre est quantifié par un paramètre d'impureté, souvent considéré comme un 
paramètre libre. 
\\

La thèse est divisée en trois chapitres. 
%
Dans le premier chapitre, nous considérons une approche de métamodélisation
unifiée afin de calculer la composition et l'équation d'état des étoiles à
neutrons froides non accrétantes pour toute fonctionnelle de matière nucléaire. 
Dans la croûte interne, l'énergie des \textit{clusters} est calculée dans 
l'approximation de la goutte liquide compressible. 
%
Dans le deuxième chapitre, nous effectuons la détermination bayésienne des
paramètres de l'équation d'état conduisant à des prédictions réalistes sur 
les observables des étoiles à neutrons froides. Nous confrontons ces 
prédictions aux contraintes des collaborations LIGO et Virgo. 
Des informations issues d'expériences de physique nucléaire, de calculs en 
théorie des perturbations chirales et d'observations astrophysiques sont 
encodées dans la fonction de vraisemblance. Les corrélations entre les 
paramètres de l'équation d'état sont explorées. La densité et pression au point 
de transition croûte-c\oe ur ainsi que le moment d'inertie de la croûte sont 
calculés à partir de la distribution \textit{posterior} des paramètres de
l'équation d'état et nous discutons l'origine des \textit{glitches} du pulsar 
de Vela. 
%
Dans le troisième chapitre, nous considérons une approche d'équilibre 
statistique nucléaire pour modéliser la croûte des protoétoiles à neutrons à 
température finie. La composition d'équilibre au point de cristallisation 
est calculée et le paramètre d'impureté, qui est une donnée importante dans 
les simulations de refroidissement des étoiles à neutrons, est évalué pour 
différentes fonctionnelles réalistes. La présence de noyaux de masse impaire et 
de charge impaire dans la croûte externe fait également l'objet d'une étude. 

\section{Structure et équation d'état des étoiles à neutrons froides non
accrétantes}

Dans ce premier chapitre, nous avons d'abord évalué l'état fondamental de la 
croûte externe que nous obtenons par application de la méthode proposée par 
Baym, Pethick et Sutherland en 1971~\cite{BPS} en utilisant les plus récentes
données sur les masses expérimentales\cite{Huang2017,Welker2017} complétées par 
des tables de masses théoriques Hartree-Fock-Bogoliubov~\cite{Samyn2002}, 
basées sur la théorie de la fonctionnelle de la densité. 
Nous avons montré que la composition de la croûte externe (et donc l'équation 
d'état) est entièrement déterminée par les connaissances actuelles sur les 
masses expérimentales jusqu'à $n_B \approx 3\times 10^{-5}$ fm$^{-3}$.
À des densités plus élevées, nous avons observé la persistance de $N=82$ pour 
chacun des quatre modèles de masse considérés : HFB-24, HFB-26, HFB-14 et FRDM. 

Nous avons proposé une version du modèle de la goutte liquide compressible
basée sur la technique de métamodélisation 
\cite{Margueron2018a,Margueron2018b}. Nous avons vu que le métamodèle offre la 
possibilité de reproduire toute fonctionnelle de matière nucléaire très 
précisément, avec une unique forme fonctionnelle. La paramétrisation de 
la tension de surface a été suggérée par des calculs microscopiques dans le 
régime des neutrons libres~\cite{Ravenhall1983} et les paramètres de surface et 
de courbure sont ajustés sur les masses expérimentales. Nous avons utilisé 
notre modèle de la goutte liquide compressible pour calculer l'état fondamental 
de la croûte interne qui est obtenu en minimisant la densité d'énergie de la 
matière à densité baryonique constante sous la condition de neutralité 
de la charge globale. Nous avons ainsi abouti à un système de quatre équations 
différentielles couplées, correspondant à des conditions d'équilibre, que nous 
avons résolues numériquement en utilisant la méthode de 
Broyden~\cite{Broyden1965}.
%
La composition de la croûte interne dans l'état fondamental, pour les 
paramètres empiriques de BSk24, a été présentée et un très bon accord avec des 
approches plus microscopiques a été observé concernant la valeur de 
$Z \approx 40$. Nous avons également vu que la fraction de protons diminue avec 
la densité de façon monotone, comme c'est le cas dans la croûte externe. 
L'équation d'état de la croûte interne a été calculée et une corrélation 
positive avec le paramètre empirique $E_{sym}$ a été révélée. Nous avons montré 
que les nombres magiques, qui disparaissent dans le cadre de l'approche
classique de la goutte liquide compressible, peuvent être retrouvés en ajoutant 
de manière perturbative des corrections de couches des protons calculées par la 
méthode de Strutinsky. De cette façon, un très bon accord avec les calculs 
Thomas-Fermi étendu avec corrections de couches Strutinksy a été observé pour 
les functionelles BSk récentes~\cite{Pearson2018}. Avec notre modèle de la
goutte liquide compressible, nous avons exploré la présence de phases 
non sphériques dans les couches profondes de la croûte interne. Nous avons 
trouvé la séquence sphères $\rightarrow$ cylindres $\rightarrow$ plaques
$\rightarrow$ tubes pour chacun des quatre modèles considérés : BSk24, SLy4, 
BSk22 et DD-ME$\delta$. Nous avons montré que la localisation du point de 
transition vers la matière homogène est très sensible au comportement de la 
tension de surface pour des valeurs extrêmes de l'isospin, contrôlé par le 
paramètre de surface isovecteur $p$. 
%
Malheureusement, il est impossible d'accéder à la valeur de ce paramètre à 
partir des données empiriques actuelles de physique nucléaire 
qui sont limitées à $I \lesssim 0.3$. Nous avons vu que 
les résultats de la littérature sur la densité et la pression de transition 
croûte-c\oe ur pour la dynamique spinodale~\cite{Ducoin2011} sont globalement 
bien reproduits par notre calcul depuis la croûte avec $p \approx 3$. Nous 
avons également confirmé la corrélation entre la densité de transition $n_t$ et 
le paramètre empirique $L_{sym}$, déjà observée dans de précédents travaux.

Nous avons déterminé les équations d'équilibre caractérisant l'état fondamental 
de la matière dans le c\oe ur \textit{npe$\mu$} jusqu'à $n_B\approx 2n_{sat}$. 
La forte corrélation entre l'énergie de symétrie et la fraction de proton a été 
expliquée. 
Nous avons constaté que les muons apparaissent à approximativement 
$n_B = 0.12$ fm$^{-3}$ pour chaque modèle. Comme dans de précédents travaux 
\cite{Wiringa1988,Douchin2001}, nous avons extrapolé la matière 
\textit{npe$\mu$} à des densités plus élevées puisque les interactions 
hypéron-hypéron et hypéron-nucléon restent actuellement très peu connues.

Enfin, nous avons souligné que l'utilisation d'une équation d'état unifiée 
est essentielle pour déterminer les observables de la croûte. En ce sens, nous 
avons proposé une métamodélisation de l'équation d'état pour une étoile à 
neutrons froide non accrétante où la croûte et le c\oe ur sont traités de 
manière uniforme, c'est-à-dire avec le même jeu de paramètres empiriques. 
En utilisant les paramètres empiriques de SLy4, nous avons montré que la 
différence relative avec l'équation d'état de Douchin et Haensel est inférieure 
à $10\%$ dans la croûte interne et à $1\%$ dans le c\oe ur.

\section{Inférence bayésienne des observables des étoiles à neutrons}

Dans ce chapitre, nous avons d'abord introduit les équations de base de
l'équilibre hydrostatique en relativité générale pour les étoiles sphériques 
non rotatives. Nous avons résolu ces équations pour différentes équations 
d'état connues afin d'obtenir la relation entre la masse et le rayon de 
l'étoile. 
Dans le même esprit, nous avons calculé le moment d'inertie ainsi que la 
déformabilité due aux effets de marée dans l'approximation de la rotation 
lente qui devrait être valable pour la plupart des pulsars. 
La détermination de la densité et de la pression au point de transition entre
la croûte et le c\oe ur, discutée dans le premier chapitre, permet le calcul 
des observables de la croûte. Nous avons ainsi calculé son épaisseur et 
la fraction du moment d'inertie résidant à l'intérieur. Nous 
avons expliqué en détail le lien entre le moment d'inertie de la croûte et le 
phénomène de \textit{glitch} de pulsar dans le consensus actuel, où un 
\textit{glitch} est observé lors du transfert de moment angulaire depuis 
les neutrons superfluides, confinés dans la croûte interne, vers le reste de 
l'étoile~\cite{Anderson1975}. 
%
L'un des principaux problèmes exposés dans la section~\ref{sec:tov} est la 
sensibilité des résultats à l'équation d'état et plus particulièrement aux
paramètres empiriques du secteur isovecteur. Nous avons montré que les 
équations d'état considérées ne réussissent pas toutes à dépasser la 
contrainte de masse maximale, $M_{max} \gtrsim 
2M_\odot$ \cite{Demorest2010,Antoniadis2013,Cromartie2020}, favorisant une 
équation d'état \textit{stiff}. 
De plus, nos résultats tendent à indiquer que le phénomène de \textit{glitch} 
ne peut pas uniquement tirer son origine de la physique de la croûte compte 
tenu des estimations actuelles de l'effet 
d'entraînement~\cite{Andersson2012,Piekarewicz2014}.
Inversement, la récente contrainte sur le paramètre de déformabilité due aux
effets de marée $\Lambda_{1.4}$ déduit de GW170817~\cite{GW1} tend à favoriser 
les équations d'état \textit{soft}. 

Nous avons rappelé le principe de l'inférence bayésienne avant d'effectuer la  
détermination bayésienne des paramètres de l'équation d'état en utilisant la 
technique de métamodélisation~\cite{Margueron2018a}. Nous avons utilisé une 
distribution \textit{prior} uniforme pour les paramètres qui tient compte 
des contraintes empiriques actuelles. 
%
Les résultats et les analyses présentés dans ce chapitre sont proches mais ne 
sont pas identiques à nos résultats 
publiés~\cite{Carreau2019cc,Carreau2019moi}. 
En effet, une analyse bayésienne dépend de façon cruciale à la fois de la
modélisation du \textit{prior} et de celle de la fonction de vraisemblance. 
Dans ce travail de thèse, nous avons fait varier à la fois la distribution
\textit{prior} des paramètres de surface, en adoptant un protocole
d'optimisation différent de celui de~\cite{Carreau2019cc,Carreau2019moi} et en 
introduisant des termes de courbure qui ont été négligées dans l'étude 
précédente, et les filtres, en explorant les effet de l'inclusion ou non des 
prédictions de la théorie des perturbations chirales concernant la pression de 
la matière nucléaire symétrique et la matière pure de neutrons. 
Les différents résultats sont compatibles à l'intérieur des barres d'erreur 
mais des corrélations différentes sont observées. L'analyse de ces différences
nous a permis  de mieux comprendre l'effet des différentes conditions. 
%
Une analyse de sensitivité sur le point de transition croûte-c\oe ur a révélé 
que les paramètres empiriques du secteur isovecteur sont les plus importants 
pour déterminer précisément la densité $n_t$ et pression $P_t$ de transition. 
Nous avons calculé la fonction de vraisemblance en tenant compte de 
contraintes sur les observables de physique nucléaire (contraintes de basse
densité), à savoir les masses expérimentales~\cite{Huang2017} 
et les calculs en théorie des perturbations chirales pour la matière nucléaire
symétrique et la matière pure de neutrons~\cite{Drischler2016}, et sur les 
observables des étoiles à neutrons (contraintes de haute densité). 
La distribution \textit{posterior} des paramètres empiriques a été analysée. 
Nous avons observé que le filtre de basse densité est très efficace pour 
contraindre les paramètres dans le secteur isovecteur, en particulier si l'on 
impose la compatibilité avec les prédictions de la théorie des perturbations
chirales pour la pression de la matière nucléaire. 
Nous avons montré que seules les contraintes en physique nucléaire donnent lieu 
à des corrélations entre les paramètres empiriques. 
Dans le secteur isovecteur, la corrélation entre $E_{sym}$ et $L_{sym}$ est 
retrouvée. Nous avons constaté que le paramètre $L_{sym}$ est corrélé avec 
$K_{sym}$ et avons trouvé de nouvelles corrélations intéressantes lors de 
l'ajout du filtre sur la pression : $r(L_{sym},Q_{sym})=-0.55$ et 
$r(K_{sym},Q_{sym})=0.52$. 

Enfin, nous avons fait des prédictions générales sur les propriétés statiques 
et les observables de la croûte des étoiles à neutrons à partir de la 
distribution \textit{posterior} des paramètres calculée dans la 
section~\ref{sec:bayes}. Dans l'ensemble, nous avons constaté que nos 
prédictions des observables de l'étoile sont en assez bon accord avec les 
contraintes sur l'équation d'état, le rayon et la déformabilité déduites de 
l'événement GW170817~\cite{De2018,GW1}. Nous avons 
obtenu $R_{1.4}=12.88_{-0.65}^{+0.53}$ km et 
$\Lambda_{1.4} = 634_{-190}^{+204}$ avec un intervalle de confiance de $90\%$. 
%
Nous avons montré que la densité et la pression de transition sont 
corrélées avec les paramètres empiriques du secteur isovecteur, notamment avec 
$Q_{sym}$. En considérant l'incertitude expérimentale et théorique 
sur ces paramètres, nous avons estimé la densité et pression de transition
à respectivement $n_t=0.068_{-0.021}^{+0.021}$ fm$^{-3}$ et
$P_t=0.263_{-0.149}^{+0.302}$ MeV/fm$^3$ à $1\sigma$. Contrairement à nos 
travaux publiés~\cite{Carreau2019cc}, nous n'observons pas une forte influence
des paramètres de surface sur les quantités au point de transition 
croûte-c\oe ur. Ceci peut s'expliquer par un contrôle plus important de 
l'énergie de surface dans cette thèse. 
%
Nous avons observé que plus la croûte est épaisse, plus la quantité de moment 
angulaire confinée dans le superfluide, coexistant avec le réseau d'ions 
dans la croûte interne, est importante. 
La grande incertitude sur la pression de transition se reflète sur le moment 
d'inertie de la croûte. En effet, pour une étoile à neutrons de $1.4M_\odot$, 
nous avons obtenu $I_{crust,1.4}/I_{1.4}=2.89_{-1.68}^{+2.51} \%$ à $1\sigma$. 
La distribution de probabilité pour la masse et le rayon du pulsar de Vela a 
été calculée pour deux estimations différentes de l'effet d'entraînement de la 
croûte. 
Ceci nous a permis de confirmer qu'une meilleure estimation quantitative de
l'effet d'entraînement est le point clé pour associer le phénomène 
de \textit{glitch} à la physique de la croûte, même si l'incertitude sur 
l'équation d'état brouille les résultats. Nous nous sommes intéressés à la 
variation du moment d'inertie de la croûte avec la masse de l'étoile et avons 
conclu que le phénomène de \textit{glitch} démontré par certains pulsars ne 
peut pas uniquement tirer son origine de la physique de la croûte dans le cas 
où l'on considère la plus grande estimation actuelle de l'effet 
d'entraînement~\cite{Delsate2016}. 

\section{Cristallisation de la croûte des protoétoiles à neutrons}

Dans ce dernier chapitre, nous nous sommes intéressés à la modélisation de la 
croûte à température finie. 
L'approximation de plasma à un composant, dans laquelle la distribution 
attendue de noyaux est remplacé par un noyau unique déterminé à partir de la 
minimisation du potentiel thermodynamique à une condition thermodynamique 
donnée, a été considérée. 
Nous avons donné les expressions des corrections entrant en jeu dans l'énergie 
libre des ions à température finie, à savoir le terme de mouvement de 
translation du centre de masse dans le liquide, le terme de vibrations 
quantiques de point zéro dans la phase solide ainsi que la contribution de 
l'interaction coulombienne qui diffère selon la phase de la matière. 
Nous avons discuté la définition de la transition de la phase liquide à la
phase solide. Puis, nous avons inclus la distribution des ions en considérant 
un plasma à plusieurs composants à l'équilibre. 
%
Dans ce cadre, des effets de mélange non linéaires apparaissent car la 
position du centre de masse de chaque ion n'est pas limitée au seul volume de 
la cellule de Wigner-Seitz associée mais peut explorer librement le 
volume total. 
Ceci est connu dans la littérature de la physique des plasmas sous le terme 
d'\guillemotleft~entropie de mélange~\guillemotright.
Les potentiels chimiques des neutrons et des protons dans le mélange se sont 
avérés être très proches de ceux dans le plasma à un composant, ce qui montre 
qu'une implémentation perturbative de l'équilibre statistique nucléaire est 
suffisante. 
%
Nous avons vu qu'un terme de réarrangement entre dans l'expression du 
potentiel thermodynamique de part l'auto-cohérence induite par la partie 
coulombienne de l'énergie libre de l'ion. 
Nous avons proposé une approximation de ce terme pour éviter de 
résoudre un problème auto-cohérent complexe en imposant la coïncidence du noyau
le plus probable avec la solution obtenue dans la limite du plasma à un 
composant. 
Comme lors de notre étude à température nulle, nous avons choisi d'utiliser 
l'approche de la goutte liquide compressible pour modéliser l'énergie libre des
\textit{clusters} dans le régime des neutrons libres. 
L'expansion de Sommerfeld, fiable à basse température, a été employée pour 
dériver l'expression de l'énergie libre de le matière nucléaire ainsi que 
du potentiel chimique des nucléons. 
En raison des faibles valeurs de la température de cristallisation prédites 
dans la croûte, nous avons négligé l'excitation des modes de surface. 

En utilisant la table de masses expérimentales AME2016, complétée par des 
tables de masses théoriques basées sur le modèle microscopique 
Hartree-Fock-Bogoliubov, nous avons obtenu des résultats pertinents pour la 
croûte externe des protoétoiles à neutrons, c'est-à-dire à basse température et 
plus particulièrement au point de cristallisation. 
%
Nous avons calculé la température de cristallisation du plasma à un composant 
dans la croûte externe. Nous avons constaté que le fait de négliger la 
contribution anharmonique dans l'énergie libre des ions dans la phase solide 
conduit à une sous-estimation de la température de cristallisation qui 
varie d'environ $5 \times 10^7$ K, à très basse pression, à $2.8 \times 10^9$
K, au voisinage de la limite de stabilité neutron et qui présente un 
comportement discontinu en raison des effets de couches. 
Nous avons vérifié que la température de cristallisation n'est pas fortement 
dépendante du modèle de masse utilisé dans les couches profondes de la croûte 
externe, tant que des tables de masses théoriques réalistes sont utilisées pour 
compléter les données expérimentales.
%
Nous avons constaté que la composition de la croûte externe au point de 
cristallisation est très proche de celle calculée dans l'hypothèse de matière
froide catalysée et que la valeur moyenne du nombre de masse $A$ et de charge 
$Z$ dans le plasma à plusieurs composants est très proche de la solution pour
le plasma à un composant dans la croûte externe. 
À mesure que la pression et la température augmentent, nous avons montré que la 
distribution de charges s'étend, indiquant alors que l'approximation du plasma 
à un composant devient moins fiable. 
%
Nous avons calculé le paramètre d'impureté $Q_{\text{imp}}$, qui 
représente la variance de la distribution de charges, à la température de 
cristallisation et nous avons retrouvé les résultats de~\cite{Fantina2020}. 
D'importantes oscillations de $Q_{\text{imp}}$ ont été observées, ce qui 
suggère que la croûte externe pourrait consister en une alternance de 
couches hautement isolantes et de couches hautement conductrices. Nous 
avons constaté que le facteur d'impureté pouvait atteindre des valeurs aussi 
élevées que $\approx 50$ lorsque la distribution de charges présente un 
caractère multimodal. 
%
La fraction des noyaux de masse impaire et de charge impaire présents dans la 
croûte externe a été évaluée. Ces quantités augmentent avec la température et, 
au point de cristallisation, les noyaux impairs constituent environ $2\%$ des 
espèces de la croûte externe et contribuent à hauteur d'environ $2.4\%$ de la 
masse baryonique de la croûte externe. 

Enfin, en utilisant les fonctionnelles BSk les plus récentes, nous avons obtenu 
des résultats pertinents pour la croûte interne des protoétoiles à neutrons,
particulièrement à la température de cristallisation. 
%
Suite à notre travail à température nulle, nous avons ajouté 
les corrections de couches à l'énergie libre des \textit{clusters}. En se 
plaçant dans l'approximation de plasma à un composant, il a été observé que 
la plus grande source de dépendance au modèle de la température de 
cristallisation et composition associée provient de la partie lisse de la 
fonctionnelle nucléaire. 
En outre, nous avons vu que les effets de couches sont fortement atténués à la 
température de cristallisation dans la croûte interne, que nous avons estimée 
entre $\approx 3 \times 10^9$ K et $\approx 9 \times 10^9$ K. 
En ce qui concerne la composition au point de cristallisation, nous avons 
observé, à faible densité, des déviations par rapport à la matière froide 
catalysée. 
%
À la température de cristallisation, la composition de la croûte interne est 
dominée par les \textit{clusters} avec $Z\approx 40$, tandis que la 
largeur de la distribution de charges est d'environ $20$, proche de la limite 
de stabilité neutron, et environ $40$, près du point de transition vers la 
matière homogène. 
Cela se reflète dans le comportement du paramètre d'impureté qui de manière 
monotone augmente avec la densité baryonique moyenne jusqu'à environ $40$ dans 
les couches les plus profondes de la croûte interne. 
Nous avons également observé que l'inclusion du terme de réarrangement est 
nécessaire pour garantir la cohérence thermodynamique. 

\section{Conclusions générales et perspectives}

L'objectif principal de cette thèse a été de fournir des prédictions réalistes 
et d'étudier les sources d'incertitudes sur les observables des étoiles à 
neutrons froides isolées et des protoétoiles à neutrons chaudes. Pour ce faire,
nous avons considéré une métamodélisation unifiée pour décrire l'énergétique de 
la matière d'étoile qui permet de tenir compte des contraintes actuelles liées 
aux expériences nucléaires et des observations astrophysiques.
\\

Nous avons envisagé une approche de métamodélisation unifiée afin de 
calculer la composition et l'équation d'état de l'étoile à neutrons froide
isolée pour toute fonctionnelle de matière nucléaire en utilisant les 
paramètres empiriques associés comme seuls ingrédients. Nous avons obtenu une
erreur qui, dans le cas de l'équation d'état basée sur SLy4, est inférieure à 
$1\%$. 
Pour modéliser l'énergie des \textit{clusters} dans la croûte interne, nous 
avons proposé une version du célèbre modèle de la goutte liquide compressible 
basée sur la technique de métamodélisation, avec une paramétrisation de la 
tension de surface suggérée par des calculs Thomas-Fermi à des ratios 
d'isospins extrêmes. 
%
La composition obtenue dans l'état fondamental suit de près les résultats de 
calculs Thomas-Fermi étendu rapportés dans la littérature. Le principal 
inconvénient de l'approche de la goutte liquide compressible est que les effets 
quantiques sont perdus. Cependant, nous avons montré que les nombres magiques 
dans la croûte interne peuvent être retrouvés en ajoutant les corrections de 
couches calculées par la méthode de Strutinsky, conduisant alors à un très bon 
accord avec les résultats de calculs Thomas-Fermi étendu avec corrections de
couches Strutinsky. 
%
La même séquence de phases non sphériques dans les couches les plus profondes 
de la croûte interne a été observée pour tous les modèles considérés mais nous 
avons souligné qu'elle est sensible au comportement de la tension de surface à 
grand isospin, restant peu contrainte à ce jour. 
%
L'incertitude sur l'équation d'état induite par le traitement de l'énergie de 
surface peut être estimée à $10\%$, ce qui correspond à la différence entre 
nos résultats avec la fonctionnelle SLy4 et l'équation d'état de Douchin et
Haensel qui utilise la même fonctionnelle mais un traitement différent de 
l'énergie de surface. 
Il en va de même pour la localisation du point de transition vers 
la matière homogène $npe$ qui a été calculée pour plusieurs modèles nucléaires 
dans le sens croûte $\rightarrow$ c\oe ur. 
%
Nous avons montré que l'anticorrélation entre la pente de l'énergie de symétrie
à la densité de saturation $L_{sym}$ et la densité de transition 
croûte-c\oe ur, signalée dans de nombreux travaux antérieurs, est retrouvée si 
l'énergie de surface est seulement optimisée sur un ensemble limité de noyaux 
magiques et semi-magiques sphériques. 
Inversement, si la tension de surface est optimisée 
sur un grand nombre de données et si des termes de courbure sont ajoutés,
cette corrélation disparaît et la sensibilité à l'équation d'état est limitée
aux paramètres empiriques d'ordres élevés dans le secteur isovecteur 
($K_{sym}$, $Q_{sym}$), peu contraints à ce jour. 

Nous avons exploité le principal atout de la technique de métamodélisation, à 
savoir le fait qu'aucune corrélation artificielle n'est introduite 
\textit{a priori} entre les paramètres empiriques, pour effectuer la 
détermination bayésienne de ces derniers, conduisant à des prédictions 
réalistes des observables des étoiles à neutrons. 
Nous avons considéré \textit{a priori} une distribution uniforme pour 
les paramètres empiriques dont les limites sont compatibles avec les 
contraintes expérimentales actuelles. 
%
La fonction de vraisemblance que nous avons construite tient compte de 
récents résultats en théorie des perturbations chirales, pour la matière 
nucléaire symétrique et la matière pure de neutrons jusqu'à $0.20$ fm$^{-3}$,
ainsi que de la contrainte de masse maximale de l'étoile et des principes
physiques de base. 
Elle comprend également une probabilité dans laquelle est encodée la capacité 
du modèle de la goutte liquide compressible à reproduire les masses de 
l'AME2016. 
Nous avons montré que le fait d'imposer les contraintes 
associées aux calculs \textit{ab initio} est très efficace pour contraindre les 
paramètres empiriques dans le secteur isovecteur et que cela donne lieu à des 
corrélations entre les dérivés de l'énergie de symétrie à la densité de 
saturation.
Nous avons fait des prédictions générales pour les propriétés statiques en 
utilisant la distribution \textit{posterior} des paramètres empiriques et avons 
constaté que nos résultats sont compatibles avec les contraintes des 
collaborations LIGO et Virgo déduites de l'événement GW170817. 
%
La seule hypothèse de la technique de métamodélisation étant la 
possibilité d'étendre l'équation d'état en une série de Taylor, ce résultat 
implique que nous n'avons pas de preuve irréfutable qu'une transition de 
phase de premier ordre ait lieu dans le c\oe ur des étoiles à neutrons, bien 
que cela ne puisse évidemment pas être exclu. 
%
Le moment d'inertie de la croûte, qui est fortement corrélé avec la 
localisation du point de transition croûte-c\oe ur, a été calculé et les
résultats tendent à indiquer que le phénomène de \textit{glitch} ne peut pas 
uniquement tirer son origine de la physique de la croûte. 
Cela ouvre des possibilités intéressantes concernant la pertinence des 
composants superfluides dans le c\oe ur, en particulier dans le canal $^3P_2$ 
pour les paires $nn$ et $pp$.

Dans la continuité d'un travail récent sur la croûte externe, nous avons 
envisagé un équilibre statistique complet des ions dans la croûte à température 
finie permettant la présence d'un gaz de neutrons. Nous avons évalué 
l'abondance des noyaux de masse impaire et des noyaux de charge impaire 
présents dans la croûte externe à la température de cristallisation. 
Leur présence est intéressante car elle pourrait causer des transitions de
phase ferromagnétiques.
%
Une dépendance en température des corrections de couches des protons dans le 
régime de la croûte interne a été implémentée. La température de 
cristallisation et la composition associée ont été calculées dans 
l'approximation de plasma à un composant et nos résultats suggèrent que la 
source la plus importante de dépendance au modèle provient de la partie lisse 
de la fonctionnelle, l'ingrédient le plus important à fixer pour la 
prédiction quantitative des propriétés de la croûte interne étant la tension de 
superface à des ratios d'isospins extrêmes. 
%
Des déviations par rapport à la matière froide catalysée ont été observées à 
faible densité au point de cristallisation, ce qui pourrait avoir un 
impact sur les simulations du processus $r$. 
Enfin, nous avons calculé le paramètre d'impureté de manière systématique dans
la croûte interne à la température de cristallisation pour quatre 
fonctionnelles BSk récentes qui tiennent compte des incertitudes actuelles sur 
l'énergie de symétrie. 
Il s'agit, à ce jour, du premier calcul du facteur d'impureté dans la croûte
interne basé sur des fonctionnelles nucléaires réalistes. Ce dernier a permis
de montrer que la contribution des impuretés est non négligeable, ce qui 
pourrait modifier les propriétés de transport dans la croûte.
\\

Il est clair que la détermination expérimentale des paramètres empiriques
d'ordres élevés du secteur isovecteur dans les prévisions théoriques de 
la théorie des perturbations chirales à basse densité et dans la modélisation
microscopique de l'énergie de surface à des ratios d'isospins extrêmes 
sont nécessaires pour réduire les incertitudes sur les observables de la croûte 
des étoiles à neutrons. 
%
De plus, nous avons montré que les contraintes sur les propriétés de la matière 
dense peuvent être déduites des observations astrophysiques dans un cadre 
bayésien. 
De nombreuses nouvelles mesures sont attendues dans un futur proche de la part
de NICER et de LIGO/Virgo. 
Nous pouvons donc espérer réduire les incertitudes sur les dérivés de l'énergie 
de symétrie d'ordres élevés et sur les observables de l'étoile en utilisant une 
méthode bayésienne ou des méthodes d'apprentissage automatique. 

Diverses applications du formalisme introduit pour la description d'un plasma à
plusieurs composants dans un équilibre statistique complet peuvent être 
envisagées à l'avenir. 
Par exemple, il serait intéressant d'étudier la présence des hypérons dans la 
croûte au point de cristallisation. 
Les géométries non sphériques pourraient également être prises en compte dans 
ce traitement, ce qui permettrait l'évaluation du paramètre d'impureté dans les 
couches profondes de la croûte. Il serait ainsi possible de vérifier 
l'hypothèse de~\cite{Pons2013} selon laquelle la présence d'une couche 
hautement résistive dans la croûte interne pourrait conduire à une limite plus 
élevée de la période de rotation des pulsars X.

Durant cette thèse, j'ai écrit une bibliothèque \textit{open source} en 
langage C, NSEoS, dans le but de fournir des outils utiles liés à la physique 
des étoiles à neutrons~\cite{NSEoS}. Cette bibliothèque a été utilisée pour 
produire tous les résultats présentés dans cette thèse. 
En outre, elle a été utilisée par différents étudiants pendant leur stage 
sous la supervision conjointe de ma directrice de thèse et de moi-même et 
nous pensons qu'elle peut servir de base à de futures études.
