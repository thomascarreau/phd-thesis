% *****************************************************************************
%     sections/chapter2.tex
%
% Last edit: 12/03
% *****************************************************************************

\chapter{Bayesian inference of neutron star observables} % 1

\section{From the equation of state to neutron star observables} % 0.75

% hydrostatic equilibrium

\subsection{Mass and radius} % 4

% derivation of the tov equation (historical approach)
% the integration of the tov equation throughout the star gives mass and radius
% it is required to integrate up to ~8nsat -> mm ok up to 2nsat...
% importance of the crust eos to properly predict radii
% results for bsk24, sly4, bsk22, ddmed
%   -> radius
%   -> maximum mass
%   -> ...
% crust-core transition point is very important in order to evaluate the crust
%   thickness/mass
% a word concerning alternative theories of gravity (rainbow, f(R), etc.)

\subsection{Moment of inertia} % 3

% explain why are we interested in the determination of this observable
% maybe compare approximations to the complete integration of the equation
% results
% crust moment of inertia ('we will see in 2.3.3 that it is linked to blabla')
% results

\subsection{Tidal deformability} % 3.5

% observable of interest since gw170817 event
% explain tidal deformability (dimless, lambda1-lambda2, lambda tilde...)
% derive equations from k2 love number
% a word about I-love-Q relations?

\section{Bayesian modeling of the equation of state} % 0.75

\subsection{Principle of Bayesian inference} % 1.5

% expressions
% discussions

\subsection{Prior distribution from nuclear experiments} % 2

\subsection{Sensitivity analysis} % 3
% ... on the crust-core transition point?

\subsection{Filters} % 0.25

\subsubsection{Low-density filters from ab initio calculations} % 2.5

% discuss ab initio calculations (saturation properties, max is 0.2, etc.)
% should we consider the pressure as a filter or not? (bsk24 does not fit in
%   the band)
% +/- 10% to account for other ab initio calculations

\subsubsection{High-density filters from astrophysical observations and 
physical requirements} % 1.5

% symmetry energy positive is a debatable hypothesis (see remark of referee)
% maximum mass as a sharp filter? 1.97? 2.23?
% more constraints in the future with nicer and lvc

\subsection{Posterior distribution of empirical parameters} % 3

% discuss the importance of filters?

\section{General predictions for neutron star observables} % 0.75

% should i add a subsection 'Posterior distribution of NS observables'?

\subsection{Confrontation with GW170817 event} % 1

\subsubsection{Equation of state} % 1

\subsubsection{Mass-radius relation} % 1.5

% speed of sound expansion allows for phase transition, unlike the mm
% -> lower radius found in 1908.10352 could be the sign of a pt to quarks?

\subsubsection{Tidal deformability} % 2

% was measured for the first time for gw170817 event allowing to calculate the
%   m1-m2, lambda1-lambda2 relations

\subsection{Bayesian analysis of the crust-core transition} % 4

% surface figures (prior/posterior)
% corr(Pt,Ksym) (link with chapter 1)
% correlation with isovector bulk parameters is softened if one consider p as a
%   free parameters (see corr matrix)

\subsection{Crustal moment of inertia and pulsar glitches} % 4

% explain pulsar glitches phenomena
% give the hypothesis explaining the phenomena
% superfluid neutrons in the crust
% is the crust enough? -> bayesian approach to consider every models (not only
%   relativistic or skyrme functionals as two separate families)
% the crust moment of inertia is calculated following eq given in 2.1.2. the
%   calculation is stopped at the crust-core interface
% report results

\section{Conclusion} % 1

% word on machine learning
