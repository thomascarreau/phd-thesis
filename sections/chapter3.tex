\chapter{Crystallization of the crust of a non-accreting neutron star}

\section{Model of the crust at finite temperature}

% in this study, no proton drip nor pasta phases are considered in the
%   free neutron regime. one can assume that proton drip does not occur in the
%   range of temperature considered here. Pasta phases are difficult to account
%   for within the mcp treatment

\subsection{One-component Coulomb plasma approximation}

At zero temperature, WS cells are supposed to be identical, thus the OCP  
(single-nucleus) approximation~\cite{Baus1980}, which considers a unique 
nucleus $(A,Z)$ for a given thermodynamic condition of pressure and temperature 
$(P,T)$, is exact. Let us recall that the equilibrium composition is 
determined by minimizing the Gibbs free energy per nucleon at fixed pressure 
(or equivalently the free energy density at fixed baryon density).
At finite temperature in the OCP approximation, the expected distribution of 
nuclei is replaced by the equilibrium nucleus obtained from the minimization of 
the relevant thermodynamic potential.
The physical properties of the OCP are fully characterized by the so-called 
dimensionless Coulomb coupling parameter,
%
\begin{equation}
  \Gamma = \frac{Ze^2}{a_N k_B T},
\end{equation}
%
where $a_N=(4\pi n_N/3)^{-1/3}$ is the ion-sphere radius, with $n_N$ the ion
density.
More precisely, $\Gamma$ allows to quantify the nonideality of the system, that
is the importance of the many-body interactions in the OCP. The lower the
temperature, the more coupled are the ions. The crystallization of the OCP into 
a lattice is observed near $\Gamma \approx 175$.

The total free energy per ion in the crust is given by
%
\begin{equation}
  F = F_{i,e} + F_g + F_e,\label{eq:fperion}
\end{equation}
%
where $F_{i,e}$ is the ion free energy in the e-cluster representation 
(see~\ref{subsec:nucenergy}), $F_g$ is the neutron gas free energy, and $F_e$ 
is the electron gas free energy, given by~\cite{Haensel2007}
%
\begin{equation}
  \mathcal{F}_e = \frac{F_e}{V_{WS}} = \varepsilon_e -
  \frac{P_r}{6}t_r^2x_r\gamma_r,
\end{equation}
%
with $t_r = T/(m_e c^2)$. The expressions of the energy density 
$\varepsilon_e$, relativistic unit of the electron pressure $P_r$, relativity
parameter $x_r$, and $\gamma_r$ are given in~\ref{subsubsec:elgas}.
Let us notice that in the regime of the outer crust, neutrons are still bound
to the nuclei, thus $F_g=0$ MeV, and consequently $F_{i,e}$ coincides with the 
ion free energy in the r-cluster representation $F_i$.
The free energy per ion reads
%
\begin{equation}
  F_{i,e} = M_{i,e} c^2 + F_i^{\text{id}} + F_i^{\text{int}},\label{eq:fie}
\end{equation}
%
where the ion mass in e-cluster representation $M_{i,e}$ has been introduced.
In Eq.~(\ref{eq:fie}), $F_i^{\text{id}}$ represents the ideal contribution to
the ion free energy, that is the noninteracting part, and $F_i^{\text{int}}$
the interacting part.

In the following, we give the expressions associated to the ideal and 
interacting contributions to the ion free energy which differ according to the 
phase of matter, either solid or liquid.
The modeling of $M_{i,e}$ as well as of the neutron gas free energy $F_g$ is 
presented in detail in~\ref{subsec:freeenfunctional}. 

\subsubsection{OCP in a liquid phase}

% blabla liquid

% derivation translational (see [GR15])

Blabla.

\subsubsection{OCP in a solid phase}

% blabla solid

% discuss in detail the corrections

Blabla.

\subsection{Multicomponent Coulomb plasma}

% the ocp approximation is expected to be reliable at low temperature

\subsubsection{Nuclear statistical equilibrium}

\subsubsection{Evaluation of the chemical potentials}

\subsubsection{Evaluation of the rearrangement term}

\subsection{Free energy functional}\label{subsec:freeenfunctional}

\subsubsection{Thermodynamics of nuclear matter}

\subsubsection{Surface plus curvature free energy}

\section{Outer crust at crystallization}

\subsection{Crystallization temperature}

\subsection{Equilibrium composition of the MCP}

\subsection{Impurity parameter}

\subsection{Abundancies of odd nuclei}

\section{Inner crust at crystallization}

\subsection{Study of the influence of shell effects in the OCP approximation}

\subsubsection{Temperature dependence of shell corrections}

\subsubsection{Equilibrium composition at crystallization for modern BSk 
functionals}

\subsection{Equilibrium composition of the MCP and the importance of the
rearrangement term}

\subsection{Dependence of the impurity parameter on the EoS}

\section{Conclusion}
