\chapter*{General introduction}
\addstarredchapter{General introduction}

% > NS formation (astro)
%   > CCSN
%   > Cooling of PNS (crystallization)
%   > NS as pulsars and observations (glitches, MM era)
% 
% > Microphysics in (P)NS (nuc)
%   > EoS (core)
%     > NN interaction (Skyrme, RMF - DFT, ab initio)
%     > SNA
%     > NSE
%   > TOV
%
% > Organization of the thesis

\section*{Neutron star formation}

Supernovae (SN) are among the most violent events in nature. 
% describe the event
%
SN have been classified into different types, depdending on their 
spectroscopic properties. 
% short descriptions of the types
%
In the consensus view, NS are final products of core collapse supernovae 
(CCSN).
% phrase protoneutron stars
It should be noticed that NS can also originates from accretion induced 
collapse, though the fraction of NS formed in this way is expected to be 
small, cite Fryer1999.

\subsection*{Core collapse supernova}

% stellar evolution

% describe ccsn dynamics (also mention chandra limit)

% observations of sn: sn1987a

% simulations of supernova explosion are very challenging

\subsection*{Protoneutron stars and neutron stars}

% supernova remnant is a pns

% pns will ultimately lead to a ns or a bh if its mass is larger than the NS
% maximum mass corresponding to TOV limit, which is sensitive to the nuclear
% EoS

\subsection*{Neutron star observations}

\section*{Microphysics of neutron stars}

\section*{Organization of the thesis}

\clearpage\thispagestyle{empty}
